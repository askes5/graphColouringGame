\chapter{Dominating sets}

We begin by listing some definitions.

\begin{definition}
    The Dominating set, $D$, of a graph $G=(E,V)$ is any subset of $V$ such that every vertex in $V$ is adjacent to at least one vertex in $D$.
\end{definition}

\begin{definition}
    The Dominating number, $\gamma(G)$, of a graph, $G=(E,V)$, is the size of the smallest dominating set of $G$.
\end{definition}

\begin{definition}
    Independent set, maximum independent set, independence number $\alpha(G)$
\end{definition}

\section{min size dominating set}

\begin{lemma}
    Let $G$ be a graph. 
        
    \[\gamma(G) \geq \alpha(G)\]
\end{lemma}

\begin{proof}
    Let $X$ be a minimum dominating set in some graph $G=(V,E)$. By definition of dominating set vertex in $V$ is adjacent to at least one vertex in      
\end{proof}
    
Recall that $\chi(G)$ is the chromatic number of the graph $G$.

\begin{theorem} [Willis 3.1 \cite{Willis2011BoundsFT}] \label{willis3.1}
    For any graph $G = (V,E)$ 

    \[\alpha(G) \leq \frac{ \left | {V} \right |}{\chi(G)}\]
\end{theorem}

Recall that $\Delta(G)$ is the maximum degree of any vertex in $G$.

\begin{theorem} [Balakrishnan 10.3.2 \cite{balakrishnan2012}] \label{balakrishnan10.3.2}
    For any graph $G$ with $n$ vertices, 
    
    \[ \left\lceil {\frac{n}{1+\Delta(G)}} \right\rceil \leq \gamma(G) \leq n - \Delta(G)\]    
\end{theorem}
    
    The trivial lower bound for $\gamma_g(G)$ when $\gamma(G)$ is known is $\gamma(G) > \gamma_g(G)$. This is because there is no dominating set smaller than $\gamma(G)$. 
    
\begin{theorem}[Ore 1962 \cite{oysteinore1962}] \label{oreDomUpper} 
    For any graph $G$ with $n$ vertices, 
    
    \[\gamma(G) \leq \frac{n}{2}\]
\end{theorem}

%TODO prove that Ore is tight

\begin{theorem}\label{gameupper}
    Let $G$ be a graph. If $x$ is a tight upper bound for the domination number, $\gamma(G)$, then  
    
    \[x \leq  \gamma_g(G)\]
\end{theorem}

\begin{proof}
    Let $G$ be a graph where $\gamma(G) = x$.
    Thus for $G$ we are unable to find a dominating set with $ < x$ vertices.
    Therefore there cannot be a winning strategy for Alice with $< x$ vertices.
    Therefore $\gamma_g(G) \geq x$
\end{proof}

%add table of classes of graphs

\begin{theorem} \label{gameDomNumLowerBound}
    Let $G$ be a graph with $n$ vertices, such that $n \geq 4$. Then,
    
    \[ \gamma_g(G) \geq \left \lfloor{\frac{n}{2}}\right \rfloor \]
    
\end{theorem}

\begin{proof}
    By combination of theorems \ref{oreDomUpper} and \ref{gameupper} we get  $ \gamma_g(G) \geq \left \lfloor{\frac{n}{2}}\right \rfloor $
\end{proof}

%TODO insert disucssion
Thereom \ref{gameDomNumLowerBound} is also proved in Alona, Baloghc, Bollobas, and Szabo 2002 \cite{AlBABoSz2002}.

The trivial upper bound is $n$. This is because the set of vertices $V(G)$ is a dominating set.

\begin{theorem}
    Let $G$ be a graph with $n$ vertices. Then,
    
    \[ \gamma_g(G) \leq \left\lceil \frac{2n}{3} \right\rceil\]
\end{theorem}
 
\begin{proof}
    A dominating set on a spanning tree in a dominating set in the parent graph.
    Thus for any graph, $G$, it suffices to show we have a winning strategy for a spanning tree of $G$. 
    let $T$ be a spanning tree of $G$.
    The winning strategy for Alice is the greedy strategy as follows. 
    
    Let $D$ be the current dominating set in $T$ i.e. neighbours of all selected vertices.
    \begin{enumerate}
        \item Pick any vertex, $v$, not in $D$ with a maximal number of neighbours not in $D$. That is maximise the set $\{x: x \in N(v) \bigwedge v \notin D\}$.
        \item repeat until you have a dominating set.
    \end{enumerate}

       worst case path graph requires twice the minimum of the path graph???
       
       with no opponent this will give n/3 thus at worst with the opponent it will take 2n/3 
       
    At worst Alice will add two vertices to 
    %TODO finish this proof
\end{proof}

\begin{theorem} %TODO prove this theorem
    Given $p$ players then,
    \[\gamma_{gp}(G) \geq p\gamma(G) \]
    
    \[\gamma_{gp}(G) \leq p \gamma_{g2}(G) \leq p\left\lceil \frac{2n}{3} \right\rceil\]

\end{theorem}













