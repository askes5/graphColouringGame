\chapter{Dominating sets}

We begin by listing some definitions.

\begin{definition}
    The Dominating set, $D$, of a graph $G=(E,V)$ is any subset of $V$ such that every vertex in $V$ is adjacent to at least one vertex in $D$.
\end{definition}

\begin{definition}
    The Dominating number, $\gamma(G)$, of a graph $G=(E,V)$ is the size of the smallest dominating set of $G$.
\end{definition}

\begin{definition}
    Independent set, maximum independent set, independence number $\alpha(G)$
\end{definition}

\section{min size dominating set}

\begin{lemma}
    Let $G$ be a graph. 
        
    \[\gamma(G) \geq \alpha(G)\]
     
\end{lemma}

\begin{proof}
    
    Let $X$ be a minimum dominating set in some graph $G=(V,E)$. By definition of dominating set vertex in $V$ is adjacent to at least one vertex in      
   
\end{proof}
    
Recall that $\chi(G)$ is the chromatic number of the graph $G$.

\begin{theorem} [Willis 2011 3.1] 
For any graph $G = (V,E)$ \cite{Willis2011BoundsFT}

    \[\alpha(G) \leq \frac{ \left | {V} \right |}{\chi(G)}\]

\end{theorem}

Recall that $\Delta(G)$ is the maximum degree of any vertex in $G$.

\begin{theorem} [Balakrishnan 2012 10.3.2] \label{balakrishnan201210.3.2}
    For any graph $G$ with $n$ vertices, 
    
    \[ \left\lceil {\frac{n}{1+\Delta(G)}} \right\rceil \leq \gamma(G) \leq n - \Delta(G)\]    \cite{balakrishnan2012}
    
\end{theorem}

\begin{theorem} \label{minDomSize}
    Let $G$ be a graph with $n$ vertices, such that $n \geq 4$. Then,
    
    \[ \gamma_g(G) > \left \lfloor{\frac{n}{2}}\right \rfloor \]
    
\end{theorem}

\begin{proof}
Worst case is $G$ is minimally connected, i.e. $G$ is a tree. Thus by \ref{balakrishnan201210.3.2} 
\[\gamma(G) \geq \left\lceil {\frac{n}{1+\Delta(G)}} \right\rceil\]       
\end{proof}

\begin{theorem}
    Let $G$ be a graph with $n$ vertices. Then,
    
    \[ \gamma_g(G) \leq n\]
\end{theorem}
