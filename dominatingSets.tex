\chapter{Dominating sets}

We begin by listing some definitions.

\begin{definition}
    The Dominating set, $D$, of a graph $G=(E,V)$ is any subset of $V$ such that every vertex in $V$ is adjacent to at least one vertex in $D$.
\end{definition}

\begin{definition}
    The Dominating number, $\gamma(G)$, of a graph $G=(E,V)$ is the size of the smallest dominating set of $G$.
\end{definition}

\begin{definition}
    Independent set, maximum independent set, independence number $\alpha(G)$
\end{definition}

\section{min size dominating set}

\begin{lemma}
    Let $G$ be a graph. 
        
    \[\gamma(G) \geq \alpha(G)\]
     
\end{lemma}

\begin{proof}
    
    Let $X$ be a minimum dominating set in some graph $G=(V,E)$. By definition of dominating set vertex in $V$ is adjacent to at least one vertex in      
   
\end{proof}
    
Recall that $\chi(G)$ is the chromatic number of the graph $G$.

\begin{theorem} [Willis 2011 Theorem 3.1] 
For any graph $G = (V,E)$ \cite{Willis2011BoundsFT}

    \[\alpha(G) \leq \frac{ \left | {V} \right |}{\chi(G)}\]


\end{theorem}

\begin{theorem} \label{minDomSize}
    Let $G$ be a graph, such that the number of vertices in G, $n$, is $\geq 4$. Then for any G,
    
    \[ \gamma_g(G) \geq \left \lceil{\frac{n}{2}}\right \rceil \]
    
\end{theorem}

\begin{proof}
asd    
\end{proof}
