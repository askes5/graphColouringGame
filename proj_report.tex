%% $RCSfile: proj_report_outline.tex,v $
%% $Revision: 1.3 $
%% $Date: 2016/06/10 03:41:54 $
%% $Author: kevin $

\documentclass[11pt
              , a4paper
              , twoside
              , openright
              ]{report}


\usepackage{float} % lets you have non-floating floats

\usepackage{url} % for typesetting urls

\usepackage{amsthm} % for definging theorems and proofs
\usepackage{amsmath}
\usepackage{amssymb}

\newtheorem{theorem}{Theorem}[chapter]
\newtheorem{claim}[theorem]{Claim}
\newtheorem{lemma}[theorem]{Lemma}
\theoremstyle{definition}
\newtheorem*{definition}{Definition}

%
%  We don't want figures to float so we define
%
\newfloat{fig}{thp}{lof}[chapter]
\floatname{fig}{Figure}

%% These are standard LaTeX definitions for the document
%%                            
\title{Graph Algorithms with Hostile Partners}
\author{Matthew Askes}

%% This file can be used for creating a wide range of reports
%%  across various Schools
%%
%% Set up some things, mostly for the front page, for your specific document
%
% Current options are:
% [ecs|msor|sms]          Which school you are in.
%                         (msor option retained for reproducing old data)
% [bschonscomp|mcompsci]  Which degree you are doing
%                          You can also specify any other degree by name
%                          (see below)
% [font|image]            Use a font or an image for the VUW logo
%                          The font option will only work on ECS systems
%
\usepackage[image,sms]{vuwproject}
\otherdegree{Bachelor of Science with Honours in Mathematics}

% You should specifiy your supervisor here with
\supervisor{Rod Downey}
% use \supervisors if there is more than one supervisor

% Unless you've used the bschonscomp or mcompsci
%  options above use
%   \otherdegree{OTHER DEGREE OR DIPLOMA NAME}
% here to specify degree

% Comment this out if you want the date printed.
\date{}

\begin{document}

% Make the page numbering roman, until after the contents, etc.
\frontmatter

%%%%%%%%%%%%%%%%%%%%%%%%%%%%%%%%%%%%%%%%%%%%%%%%%%%%%%%

%%%%%%%%%%%%%%%%%%%%%%%%%%%%%%%%%%%%%%%%%%%%%%%%%%%%%%%

\begin{abstract}

A short description of the project goes here.

\end{abstract}

%%%%%%%%%%%%%%%%%%%%%%%%%%%%%%%%%%%%%%%%%%%%%%%%%%%%%%%

\maketitle

\include{acknowledge}

\tableofcontents

% we want a list of the figures we defined
\listof{fig}{Figures}

%%%%%%%%%%%%%%%%%%%%%%%%%%%%%%%%%%%%%%%%%%%%%%%%%%%%%%%

\mainmatter

%%%%%%%%%%%%%%%%%%%%%%%%%%%%%%%%%%%%%%%%%%%%%%%%%%%%%%%

% individual chapters included here
%\include{introduction}
%\include{using}
%\include{example}
%\chapter{Conclusion}\label{chpt:con}

%todo inthis report we have ...

%todo there are many more game [list a few with references] loved to include but would drag out the report

%summary of what we have done

\section{Summary}

In this report we have studied a wide variety of different games and strategies. In our study of the dominating game we found both upper and lower bounds the game dominating number. The upper bound was found by the way of a simple strategy for Alice. The lower bound was found by exploring a property of tight bounds for the dominating number. For the class of trees we demonstrated two different strategies for Alice and their relation to the $3/5$--conjecture (conjecture \ref{conj:treedomup}). We concluded our study of the dominating game with a look at the $(a,b)$ variant. We managed to extend a previous proof to get a new upper bound of for the $(a,b)$--dominating game. 
Concluding the chapter on the dominating game we studied of the independent dominating game. While slightly less in-depth than the other sections, it touched on most areas in the literature.

Our look at the colouring game was divided into two main parts; lower bounds for the colouring game and upper bounds by the way of the marking game. The lower bounds for which came virtue of some new extensions to the $(a,b)$--colouring game of some already known results.
The marking game is the largest section in this report. Most of this time was spent looking at the Activation Strategy and variations thereof. As part of our studies we provided two new proofs for graphs of bounded treetree and pathwitdh. We then showed how that activation strategy can be modified for the $(a,b)$--marking game. We concluded our study of the marking with a brief look at the refined activation strategy.

\section{The Colouring Game as an App}
%the game that I build and possible applictions



\section{Other Games}
%other games and problems that we didn't include
We conclude this report we take a brief look at some games that we would have loved to include but did not have the time to. These games are primed for further research. They are the total domination game, the perfect code game, and online colouring graphs of bounded pathwidth.

The \textit{total domination game} is another variation of the domination game. Alice and Bob take turns building a total dominating set $D$ in a graph $G=(V,E)$. A total dominating set is a dominating set $D$ such that every vertex in $D$ is adjancent to another vertex in $D$. So, on their turn Alice and Bob add a vertex $v$ to $D$ such that $N[D]$ increase is size and $v$ is adjacent to a vertex in $D\setminus v$. The game ends when $D$ forms a dominating set. The total dominating game is a relatively recent creation. The introductory paper \cite{henning2015total} was only published in 2015. As every total dominating set is a dominating set the total dominating game bounds the dominating game. This gives some relation between the two versions. However, the two game do differ in many ways. Exactly how the games differ is something we would have loved to include, but didn't have the time.

A \textit{perfect code} in a graph $G=(V,E)$ is an independent subset $C$ of $V$ such that every vertex in $V$ is either in $C$ or adjacent to exactly one vertex in $C$. In the \textit{perfect code game} Alice and Bob take turns adding a vertex to $C$ such that $C$ forms a partial perfect code. The game continues until $C$ forms a perfect code, or cannot be enlarged further. If at the end on the game $C$ is a perfect code then Alice wins, and Bob wins if $C$ is not a perfect code. Some graphs do not admit a perfect code. On such graphs Alice will never win. There are other graphs that admit perfect codes but on which Bob will always win. The raises the question, on which classes graphs can Alice win? We can also ask, how small can Alice force the perfect code to be? These are the sorts of questions that could motivate further research. 

When \textit{online colouring} a graph, a single vertex is revealed along with its relation to all previously revealed vertices. The revealed vertex is then coloured. How many colours do we need to online colour a graph? A graph of pathwidth $k$ can be online coloured using $3k+1$ colours \cite{kierstead1981extremal}. 
We also have, when playing the colouring game on an interval graph with width $k$ Alice will always win if the number of available colours is at least $3k+1$ (Theorem \ref{thm:pathwidth}). It is quite a coincidence that these two concepts have the same bound on the same class of graphs. It gets even weirder when you consider the fact that the bound for the colouring game was not found using the colouring game. Exactly why these bounds are the same is an open question. 















    

\chapter{Introduction}

\chapter{Dominating sets}

We begin by listing some definitions.

\begin{definition}
    The Dominating set, $D$, of a graph $G=(E,V)$ is any subset of $V$ such that every vertex in $V$ is adjacent to at least one vertex in $D$.
\end{definition}

\begin{definition}
    The Dominating number, $\gamma(G)$, of a graph $G=(E,V)$ is the size of the smallest dominating set of $G$.
\end{definition}

\begin{definition}
    Independent set, max independent set, indepence number $\alpha(G)$
\end{definition}

\section{min size dominating set}

\begin{lemma}
    Let $G$ be a graph and $X$ be a subset of the vertices of $G$. 
        
    The minimum size of a dominating set is greater than or equal to the size of the maximum independent set.
     
\end{lemma}

Recall that $\chi(G)$ is the chromatic number of the graph $G$.

\begin{proof}
    
    Let $X$ be a minimum dominating set in the graph $G$.
    
    
    
\end{proof}

\begin{theorem} [Willis 2011 Theorem 3.1] 
For any graph $G = (V,E)$ \cite{Willis2011BoundsFT}

    \[\alpha \leq \frac{|V|}{\chi(G)}\]


\end{theorem}

\begin{theorem} \label{minDomSize}
    Let $G$ be a graph, such that the number of vertices in G, $n$, is $\geq 4$. Then for any G,
    
    \[ \gamma_g(G) \geq \left \lceil{\frac{n}{2}}\right \rceil \]
    
\end{theorem}

\begin{proof}
asd    
\end{proof}

%%%%%%%%%%%%%%%%%%%%%%%%%%%%%%%%%%%%%%%%%%%%%%%%%%%%%%%

\backmatter

%%%%%%%%%%%%%%%%%%%%%%%%%%%%%%%%%%%%%%%%%%%%%%%%%%%%%%%


%\bibliographystyle{ieeetr}
\bibliographystyle{acm}
\bibliography{sample}


\end{document}
