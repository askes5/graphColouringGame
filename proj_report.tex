%% $RCSfile: proj_report_outline.tex,v $
%% $Revision: 1.3 $
%% $Date: 2016/06/10 03:41:54 $
%% $Author: kevin $

\documentclass[11pt
              , a4paper
              , twoside
              , openright
              ]{report}


\usepackage{float} % lets you have non-floating floats

\usepackage{url} % for typesetting urls

\usepackage{amsthm} % for definging theorems and proofs
\usepackage{amsmath}
\usepackage{amssymb}

\newtheorem{claim}{Claim}
\newtheorem{lemma}{Lemma}
\newtheorem{theorem}{Theorem}
\theoremstyle{definition}
\newtheorem*{definition}{Definition}

%
%  We don't want figures to float so we define
%
\newfloat{fig}{thp}{lof}[chapter]
\floatname{fig}{Figure}

%% These are standard LaTeX definitions for the document
%%                            
\title{Graph Algorithms with Hostile Partners}
\author{Matthew Askes}

%% This file can be used for creating a wide range of reports
%%  across various Schools
%%
%% Set up some things, mostly for the front page, for your specific document
%
% Current options are:
% [ecs|msor|sms]          Which school you are in.
%                         (msor option retained for reproducing old data)
% [bschonscomp|mcompsci]  Which degree you are doing
%                          You can also specify any other degree by name
%                          (see below)
% [font|image]            Use a font or an image for the VUW logo
%                          The font option will only work on ECS systems
%
\usepackage[image,ecs]{vuwproject}
\otherdegree{Bachelor of Science with Honours in Mathematics}

% You should specifiy your supervisor here with
\supervisor{Rod Downey}
% use \supervisors if there is more than one supervisor

% Unless you've used the bschonscomp or mcompsci
%  options above use
%   \otherdegree{OTHER DEGREE OR DIPLOMA NAME}
% here to specify degree

% Comment this out if you want the date printed.
\date{}

\begin{document}

% Make the page numbering roman, until after the contents, etc.
\frontmatter

%%%%%%%%%%%%%%%%%%%%%%%%%%%%%%%%%%%%%%%%%%%%%%%%%%%%%%%

%%%%%%%%%%%%%%%%%%%%%%%%%%%%%%%%%%%%%%%%%%%%%%%%%%%%%%%

\begin{abstract}

A short description of the project goes here.

\end{abstract}

%%%%%%%%%%%%%%%%%%%%%%%%%%%%%%%%%%%%%%%%%%%%%%%%%%%%%%%

\maketitle

\include{acknowledge}

\tableofcontents

% we want a list of the figures we defined
\listof{fig}{Figures}

%%%%%%%%%%%%%%%%%%%%%%%%%%%%%%%%%%%%%%%%%%%%%%%%%%%%%%%

\mainmatter

%%%%%%%%%%%%%%%%%%%%%%%%%%%%%%%%%%%%%%%%%%%%%%%%%%%%%%%

% individual chapters included here
%\include{introduction}
%\include{using}
%\include{example}
%\chapter{Conclusion}\label{chpt:con}
\chapter{Introduction}

\chapter{Dominating sets}

We begin by listing some definitions.

\begin{definition}
    The Dominating set, $D$, of a graph $G=(E,V)$ is any subset of $V$ such that every vertex in $V$ is adjacent to at least one vertex in $D$.
\end{definition}

\begin{definition}
    The Dominating number, $\gamma(G)$, of a graph $G=(E,V)$ is the size of the smallest dominating set of $G$.
\end{definition}

\section{min size dominating set}

\begin{theorem} \label{minDomSize}
    Any graph, $G=(E,V)$, with $|V| \geq 7$ has $\gamma_g(G) \geq \left \lceil{\frac{n}{2}}\right \rceil $
\end{theorem}

\begin{proof}
asd    
\end{proof}

%%%%%%%%%%%%%%%%%%%%%%%%%%%%%%%%%%%%%%%%%%%%%%%%%%%%%%%

\backmatter

%%%%%%%%%%%%%%%%%%%%%%%%%%%%%%%%%%%%%%%%%%%%%%%%%%%%%%%


%\bibliographystyle{ieeetr}
\bibliographystyle{acm}
\bibliography{sample}


\end{document}
