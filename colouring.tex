\chapter{Colouring}

\begin{theorem}
    Let $T$ be a tree, if we have $p$ players then,
    \[\chi_g(T) \geq p + 2 \]
\end{theorem}

The following proof is an extended version of the proof of Theorem 5.4 in \cite[Bodlaender 1990]{bodlander1990}

\begin{proof}
    Consider the graph $G$ as defined in figure \ref{multiPlayerTree1}. 
          
   \begin{figure}[h]
       \centering
        \begin{tikzpicture}[scale=.9, transform shape]
        \tikzstyle{every node}=[circle]; 
        \node [normal] (v7) at (-7.5,1) {};
        \node [normal] (v6) at (-11,1) {};
        \node [normal] (v5) at (-14.5,1) {};
        \node [normal] (v2) at (-18.5,1) {};
        \node [normal] (v1) at (-21,1) {};
        \node [normal] (v3) at (-19,-1) {};
        \node [normal] (v4) at (-17,-1) {};
        \node [ normal] (v11) at (-15,-1) {};
        \node [normal] (v10) at (-13.5,-1) {};
        \node [normal] (v9) at (-11.5,-1) {};
        \node [normal] (v8) at (-10,-1) {};
        \node [normal] (v15) at (-9,-1) {};
        \node [normal] (v16) at (-8,-1) {};
        \node [normal] (v17) at (-6.5,-1) {};
        \node [normal] (v12) at (-20,-1) {};
        \node [normal] (v13) at (-16,-1) {};
        \node [normal] (v14) at (-12.5,-1) {};
        \node [normal] (v18) at (-5,1) {};
        \draw  (v1) edge (v2);
        \draw  (v1) edge (v2);
        \draw  (v2) edge (v3);
        \draw  (v2) edge (v4);
        \draw  (v2) edge (v5);
        \draw  (v5) edge (v6);
        \draw  (v6) edge (v7);
        \draw  (v6) edge (v8);
        \draw  (v6) edge (v9);
        \draw  (v5) edge (v11);
        \draw  (v5) edge (v10);
        \draw  (v2) edge (v12);
        \draw  (v13) edge (v5);
        \draw  (v14) edge (v6);
        \draw  (v15) edge (v7);
        \draw  (v16) edge (v7);
        \draw  (v7) edge (v17);
        \draw  (v7) edge (v18);
        \node at ($(v3)!.5!(v4)$) {\ldots};
        \node at ($(v3)!.5!(v4)$) {\ldots};
        \node at ($(v11)!.5!(v10)$) {\ldots};
        \node at ($(v9)!.5!(v8)$) {\ldots};
        \node at ($(v16)!.5!(v17)$) {\ldots};
        
        \draw [decorate,decoration={calligraphic brace,amplitude=11pt,mirror,raise=2ex}, line width=1.5pt]
        (v12) -- (v4) node[midway,yshift=-3em]{$p$ vertices};
        \end{tikzpicture}
        \caption{}
        \label{multiPlayerTree1}
    \end{figure}    

    
    We give a strategy for Bob with $p+1$ colours. Let the colours be $\{c_1,c_2,\ldots,c_{p+1}\}$
    On Alice's first move she picks any vertex, $v$, and colours it. Let the colour of $v$ be $c_1$.
    Bobs first move is to colour any vertex with distance 3 to $v$. We now have a subgraph in $G$ of the type shown in figure \ref{subgraphColouring}. We then colour $y_1$ \ldots $y_{p-2}$ with $c_2$ \ldots $c_{p-1}$ respectively.
\begin{figure}[h]
    \centering
    \begin{tikzpicture}[scale=1,transform shape]
    \tikzstyle{every node}=[circle]; 
    \node [label=above left:4, normalBlack, label=below right:{\footnotesize$c_1$}] (v6) at (-11,1) {};
    \node [label=above left:3, normal] (v5) at (-14,1) {};
    \node [label=above left:2, normal] (v2) at (-19,1) {};
    \node [label=above left:1, normalBlack, label=below right:{\footnotesize$c_1$}] (v1) at (-22,1) {};
    \node [label=below left:$x_1$, normal] (v12) at (-21,-1) {};
    \node [label=below left:$x_2$, normal] (v3) at (-20,-1) {};
    \node [label=below left:$x_p$, normal] (v4) at (-18,-1) {};
    \node [label=below left:$y_1$, normalBlack] (v13) at (-16,-1) {};
    \node [label=below left:$y_2$, normalBlack] (v11) at (-15,-1) {};
    \node [label=below left:$y_p$, normal] (v10) at (-13,-1) {};
    \draw  (v1) edge (v2);
    \draw  (v2) edge (v3);
    \draw  (v2) edge (v4);
    \draw  (v2) edge (v5);
    \draw  (v5) edge (v6);
    \draw  (v5) edge (v11);
    \draw  (v5) edge (v10);
    \draw  (v2) edge (v12);
    \draw  (v13) edge (v5);   
    \node (e1) at ($(v3)!.5!(v4)$) {\ldots};
    \node (e1) at ($(v11)!.5!(v10)$) {\ldots};
    \end{tikzpicture}
    \caption{}
    \label{subgraphColouring}
\end{figure}

    We consider two cases.
    
    \begin{enumerate}
        \item Alice picks $2$, $x_1$, $x_2$, \ldots, or $x_p$.
        
        Bob colours $y_{p-1}$ $c_p$ and $y_{p}$ $c_{p+1}$. 
        Vertex $3$ now has $p+1$ different coloured neighbours and thus Bob wins. 
        
        \item Alice picks $3$,$y_{P-1}$, or $y_p$
        
        Bob colours $x_1$ \ldots $x_{p-1}$ with $c_2$ \ldots $c_{p}$ respectively.
        Vertex $2$ now has $p+1$ different coloured neighbours and thus Bob wins. 
    \end{enumerate}

\end{proof}












