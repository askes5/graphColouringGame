
 
\documentclass[]%   %the handout option removes all pauses for printing
{beamer}
%\usepackage{amsmath,amssymb,latexsym}

\mode<presentation>
{
%  \usetheme{Antibes}
%  \usetheme{Bergen}
 %\usetheme{Berkeley}% might be useful for a lecture
%  \usetheme{Berlin}%  this is quite nice 	
 % \usetheme{Boadilla}%  lots of white space VIELLEICHT
  %\usetheme{Copenhagen}% quite similar to Warsaw, but perhaps preferable to it
%  \usetheme{Darmstadt}% quite similar to Warsaw, but perhaps preferable to it
%  \usetheme{Dresden}% no boxes, uses much space but could be ideal if slightly modified ????
%  \usetheme{Frankfurt}% also quite similar to Warsaw but better
%  \usetheme{Goettingen}% has much space but also large fonts, sidebar on the right
%  \usetheme{Hannover}% as Goettingen, with sidebar on the left
% \usetheme{Ilmenau}% perhaps the best in the pseudo-Warsaw group
%  \usetheme{JuanLesPins}% also pseudo-Warsaw, not optimal
%  \usetheme{Luebeck}% again quite similar to Warsaw; not bad
%  \usetheme{Madrid}% nice but font too large, no section display but frame counter!!!  ?????
%  \usetheme{Malmoe}% no boxes, but otherwise quite nice ?????
%  \usetheme{Marburg}% very similar to Goettingen, no boxes
 %\usetheme{Montpellier}% very white, large fonts, second section not shown again ????
%  \usetheme{NewcastleAU}%
%  \usetheme{PaloAlto}% very similar to Berkeley
 \usetheme{Pittsburgh}% very white, lots of space but no top- or sidebar information VIELLEICHT
%  \usetheme{Rochester}% blue bar on top without any info in it
%  \usetheme{Singapore}% very white, no boxes, all essential navigation needs, enough space
%  \usetheme{Szeged}% very white, no boxes, wastes space, all essential navigation needs
%  \usetheme{Warsaw}  %VIELLEICHT
%  \usetheme{boxes}% very white, no boxes, ample space, no section navigation
%  \usetheme{default}% very similar to boxes, not bad as a default
%  \usetheme{compatibility}% does not take all commands
% or ...
%  \usetheme{bars}%
 %\usetheme{classic}% seems to be ideal for a normal talk; just the essentials, no color overload
%  \usetheme{lined}% semi-classic, NOTE PLACEMENT OF LOGO
%  \usetheme{plain}%  does not take all commands
%  \usetheme{shadow}% what's the difference to Warsaw? Not much.
%  \usetheme{sidebar}%
%  \usetheme{split}% VIELLEICHT
%  \usetheme{tree}%

 \setbeamercovered{transparent}
  % or whatever (possibly just delete it)
}


\usepackage[english]{babel}
% or whatever

\usepackage[latin1]{inputenc}
% or whatever

\usepackage{times}
\usepackage[T1]{fontenc}
% Or whatever. Note that the encoding and the font should match. If T1
% does not look nice, try deleting the line with the fontenc.

  \usepackage{booktabs}
%  \usepackage{topcapt}
\title[A beamed example] % (optional, use only with long paper titles)
{A beamed example}

%\subtitle
%{\scriptsize Based on joint work with Udo Baumgartner and Bertrand R\'emy} % (optional)

\author[Jacqui Ramagge] % (optional, use only with lots of authors)
{Jacqui Ramagge}
%{F.~Author\inst{1} \and S.~Another\inst{2}}
% - Use the \inst{?} command only if the authors have different
%   affiliation.

\institute[U of Wollongong] % (optional, but mostly needed)
{The University of Wollongong}
%	{
%	  \inst{1}%
%	  Department of Computer Science\\
%	  University of Somewhere
%	  \and
%	  \inst{2}%
%	  Department of Theoretical Philosophy\\
%	  University of Elsewhere}
% - Use the \inst command only if there are several affiliations.
% - Keep it simple, no one is interested in your street address.

\date[CRM, June 2007] % (optional)
{Centre de Recerca Matem\`atica, June 2007}

\subject{Talks}
% This is only inserted into the PDF information catalog. Can be left
% out.

% If you have a file called "university-logo-filename.xxx", where xxx
% is a graphic format that can be processed by latex or pdflatex,
% resp., then you can add a logo as follows:
	
%	 \pgfdeclareimage[height=0.5cm]{university-logo}{blue-UoNheader}
%	 \logo{\pgfuseimage{university-logo}}

% Delete this, if you do not want the table of contents to pop up at
% the beginning of each subsection:
%	\AtBeginSubsection[]
%	{
%	  \begin{frame}<beamer>
%	    \frametitle{Outline}
%	    \tableofcontents[currentsection,currentsubsection]
%	  \end{frame}
%	}


% If you wish to uncover everything in a step-wise fashion, uncomment
% the following command:

%\beamerdefaultoverlayspecification{<+->}

\providecolor[named]{loc-dark-green}{RGB}{0,96,0}
\providecolor[named]{purple}{RGB}{80,0,20}


%#################        MACROS      ##################################

\newcommand{\ZZ}{\mathbb{Z}}%
\newcommand{\RR}{\mathbb{R}}%
\newcommand{\CC}{\mathbb{C}}%
\newcommand{\QQ}{\mathbb{Q}}%
\newcommand{\KK}{\mathbb{K}}%
\newcommand{\NN}{\mathbb{N}}%


\begin{document}

\begin{frame}%
  \titlepage
\end{frame}


\begin{frame}% # 00
\frametitle{Overview}\tableofcontents[hideallsubsections]% pausesections

\end{frame}

\section{Section One} %This will appear on overview slide

\begin{frame}[label=First-slide]
\frametitle{First Slide Title}

The usual \emph{emphasising} works.

\bigskip
You can write in colour using built-in definitions such as
{\color{blue} blue} and {\color{red} red}.

\bigskip
{\color{purple} These words} are in the purple colour I defined.

\bigskip
\pause

These words will appear in grey until you press return.

\bigskip
But
\pause
\uncover<3->{
these words will be invisible until you press return again.
}

\bigskip
\pause
These words will appear in grey until you press return 3 times, which beamer counts as the 4th reveal.


\hfill\hyperlink{sample buttons<1>}{\beamerreturnbutton{Sample Buttons}}


\end{frame}


\begin{frame}
\frametitle{More Revealing}

You can do more fancy things.

\bigskip
\pause
\uncover<2-2>{
These words will appear}
\pause
\uncover<3->{
and then go grey again.
}

\bigskip
\pause
\uncover<4-5>{
These words will be visible on the fourth reveal (ie third return), be black for two reveals and then become grey.
}

\bigskip
\pause
\uncover<4->{
These words will be visible in grey on the fourth reveal and become black after the fifth.
because the number in the uncover command is larger than the reveal stage we are currently on.}

\bigskip
\pause
This is just so you know I've finished messing around.

\end{frame}


\begin{frame}
\frametitle{Revealing Mathematics}

You can reveal mathematics in text
$y=f(x)$
\pause
or in display mode
\[
y=\int e^x\, dx
\]
and you can even
\pause
\uncover<3->{ uncover stuff like $e^{i\pi}=-1$ or
\[
\frac{d}{dx}(e^x)=e^x
\]
}

\end{frame}


\begin{frame}
\frametitle{Blocks}

\begin{block}{Theorem (somebody)}
This is a statement.
\end{block}

\medskip
\begin{proof}
Here is my proof, complete with an automatic tombstone.
\end{proof}

\medskip
\begin{exampleblock}{Examples}
This block is in a different colour.
\end{exampleblock}

\end{frame}


\section{Section Two} %This will appear on overview slide


\begin{frame}
\frametitle{Items and enumerations}

\begin{itemize}
\item Itemizing
\item works as usual
\end{itemize}

\begin{enumerate}
\item as does
\item enumeration.
\item You can still label \label{item3}
\item things
\end{enumerate}

and then refer to them as Item~\ref{item3}.


\end{frame}


\begin{frame}
\frametitle{Yet More Revealing}

\begin{itemize}
\item You can also do reveals
\pause
\item within itemized environments
\pause
\item<3-3> and even uncover
\pause
\item<4-> and hide.
\end{itemize}

\end{frame}

\begin{frame}
\frametitle{Pictures}

You can also import .jpg or .pdf files as images

\medskip
\begin{center}
\includegraphics[scale=0.75]{mp/hexagon-7}
\end{center}

\pause
and they can even be uncovered

\uncover<2->{\hfil\hfil\includegraphics<2->[scale=0.75]{mp/hexagon-9}\hfil}


\end{frame}

\begin{frame}[label=buttons]
\frametitle{Buttons}

\bigskip
You can make buttons to help you navigate through your presentation.

\bigskip
The syntax is

\bigskip
$\backslash$hyperlink\{pageref$<$reveal stage$>$\}%
\{\{$\backslash$beamerbutton\{button label\}\}\}

\bigskip
You need to have labelled the frame using
$[$frame=pageref$]$.

\end{frame}

\begin{frame}[label=sample buttons]
\frametitle{Sample Buttons}

So $\backslash$hyperlink\{First-slide$<$1$>$\}%
\{\{$\backslash$beamerbutton\{top of first slide\}\}\}
produces

\bigskip
\hyperlink{First-slide<1>}{{\beamerbutton{top of first slide}}}
\bigskip

which takes you to the top of the first slide,
\pause
whereas

\bigskip
$\backslash$hyperlink\{First-slide$<$3$>$\}%
\{\{$\backslash$beamerbutton\{first slide 3rd reveal\}\}\}

produces

\bigskip
\hyperlink{First-slide<3>}{{\beamerbutton{first slide 3rd reveal}}}
\bigskip

which should take you to the 3rd reveal in the first slide.

\end{frame}

\begin{frame}
\frametitle{Going back}

There is a return button with the syntax

\bigskip
{\small
$\backslash$hyperlink\{pageref$<$reveal stage$>$\}%
\{\{$\backslash$beamerreturnbutton\{button label\}\}\}
}

\bigskip
which produces buttons with an arrowhead like
\hfill\hyperlink{buttons<1>}{\beamerreturnbutton{Buttons}}

for when you want to go back to the main presentation if you have taken a detour.
\end{frame}

\begin{frame}[label=final detours]
\frametitle{Final Detours}

You can put minor detours after your final slide

\hyperlink{detour<1>}{{\beamerbutton{a detour}}}

\pause
and come back later

The only thing to watch out for is that you don't go
from your final slide to the detour again after the end of your talk.

\end{frame}

\begin{frame}
\frametitle{Pilot Eject Buttons}

I recommend having buttons which take you to the last slide at appropriate cut-off points during the talk but I usually don't label them so as not to get people's hopes up$\dots$

\hfill\hyperlink{last slide<1>}{{\beamerbutton{}}}

\end{frame}


\begin{frame}[label=last slide]
\frametitle{Any questions?}

\vfill
\vfill
\pause
\uncover<2->{
Thank you for your attention.}

\vfill
\vfill
\vfill
\vfill
\vfill
I sometimes include buttons linked to slides in case I get questions
\bigskip
\hyperlink{First-slide<1>}{{\beamerbutton{top of first slide}}}

\end{frame}

\begin{frame}[label=detour]
\frametitle{Detour}

Back you go!
\hfill\hyperlink{final detours<2>}{\beamerreturnbutton{Final Detours}}


\end{frame}





\end{document}


