\chapter{Graph Domination}\label{chpt:domSet}

\section{Introduction}

In chapter \ref{chpt:into} we introduced the Dinner Party Problem. To formalize the Dinner Party Problem in the language of graphs we introduce the concept of a dominating set. The Dinner Party Problem  can be thought of as  a graph game where Alice and Bob are building a dominating set in the graph of guests. A guest is in the dominating set if a platter is placed next to them. A guest is dominated if they are within arms reach of a platter. The game continues until every guest is dominated, aka feed. 

\begin{definition}[Dominating Set]
    Let $G$ be a graph. A \textit{dominating set} $D$ of $G$ is a subset of $V(G)$ such that every vertex is either in $D$ or is adjacent to at least one vertex in $D$.
\end{definition} 

\begin{definition}[Domination Number]
    Let $G$ be a graph. The Domination number $\gamma(G)$ of $G$ is the minimum size of a dominating set in $G$.
\end{definition}

%todo rewrite
The class of NP problems can be thought of as the problems which have no efficient algorithm for finding a solution. 
Finding a dominating set of size at most $k$ is a classic NP-complete problem \cite[p.~190]{garey1979computers}. This means that we believe it to have no polynomial time (read efficient) algorithm and is at least as hard as any problem in NP. Basically, this means that in general finding dominating sets can only be done by checking every possible dominating set.

One way of visualising dominating sets is to consider cell tower placement. Suppose we wish to build a 5G network in a city. We consider a graph of buildings in a city. The vertices are buildings and two vertices $u$ and $v$ are connected if a 5G tower placed on building $u$ one will provide coverage to building $v$. A set of buildings that would provide coverage to the whole city is a dominating set. The minimum number of towers needed is the domination number of this graph.

The size of a minimal dominating set is in some sense a measure of how closely connected a graph is. A graph with a low domination number is densely connected, and a graph with a high domination number is loosely connected. For example, a city that is densely packed requires fewer towers. Every new 5G tower reaches many buildings. Whereas, a sparsely populated city will require more towers. Each tower only services a few buildings.   

As a further example consider the wheel graph $W_9$ and the cycle graph $C_8$ in figures \ref{fig:wheelW9} and \ref{fig:cycleC8}.
\begin{figure} [h]
    \centering
    \begin{minipage}{.45\textwidth}
        \centering
        \begin{tikzpicture}[scale=.6]
        \node [normal] (v8) at (1,2) {};
        \node [normal] (v6) at (4,-1) {};
        \node [normal] (v4) at (1,-4) {};
        \node [normal] (v2) at (-2,-1) {};
        \node [normal] (v1) at (-1.1,1.1) {};
        \node [normal] (v7) at (3.1,1.1) {};
        \node [normal] (v5) at (3.1,-3.1) {};
        \node [normal] (v3) at (-1.1,-3.1) {};
        \draw  (v1) edge (v2);
        \draw  (v2) edge (v3);
        \draw  (v3) edge (v4);
        \draw  (v4) edge (v5);
        \draw  (v5) edge (v6);
        \draw  (v6) edge (v7);
        \draw  (v8) edge (v1);
        \draw  (v8) edge (v7);
        \node [normalBlack] (v9) at (1,-1) {};
        \draw  (v8) edge (v9);
        \draw  (v9) edge (v1);
        \draw  (v2) edge (v9);
        \draw  (v9) edge (v3);
        \draw  (v4) edge (v9);
        \draw  (v9) edge (v5);
        \draw  (v9) edge (v7);
        \draw  (v6) edge (v9);
        \end{tikzpicture}
        \caption{The wheel graph $W_9$}
        \label{fig:wheelW9}
    \end{minipage}
    \begin{minipage}{.45\textwidth}
        \centering
        \begin{tikzpicture}[scale=.6]
        \node [normalBlack] (v8) at (1,2) {};
        \node [normalBlack] (v6) at (4,-1) {};
        \node [normal] (v4) at (1,-4) {};
        \node [normal] (v2) at (-2,-1) {};
        \node [normal] (v1) at (-1.1,1.1) {};
        \node [normal] (v7) at (3.1,1.1) {};
        \node [normal] (v5) at (3.1,-3.1) {};
        \node [normalBlack] (v3) at (-1.1,-3.1) {};
        \draw  (v1) edge (v2);
        \draw  (v2) edge (v3);
        \draw  (v3) edge (v4);
        \draw  (v4) edge (v5);
        \draw  (v5) edge (v6);
        \draw  (v6) edge (v7);
        \draw  (v8) edge (v1);
        \draw  (v8) edge (v7);
        \end{tikzpicture}
        \caption{The cycle graph $C_7$}
        \label{fig:cycleC8}
    \end{minipage}
\end{figure}
The wheel has a dominating set of size 1, just the centre vertex. Whereas, the cycle graph has a dominating set of size 3. In figures  \ref{fig:wheelW9} and \ref{fig:cycleC8} the dominating sets are denoted as black vertices. This can be interpreted as the wheel graph being more closely connected than the cycle graph. And when you observe the graphs this distinction makes sense.

The \textit{dominating game} was introduced by Bo\v{s}tjan Bre\v{s}ar, Sandi Klav\v{z}ar, and Douglas F. Rall 2010 \cite{BrKlRa2010}. In this game Alice and Bob take turns adding vertices to a set until it forms a dominating set.

In this chapter we will show how the dominating game can be bounded in terms of both the number of vertices and the domination number. We also demonstrate a strategy for Alice that ensures she will win if the number of turns passes half the number of vertices.

%################################################################################

\section{The Dominating game}\label{sec:dominating_game}
For a graph $G$ and a set $X\subseteq V(G)$ we denote $N[X]$ the set of neighbours of $X$ including $X$. That is $N[X] = \{v\in V(G) : \exists_{u\in X} (u,v)\in E(G)\}\cup X$.  

We define the dominating game as follows. Let $G$ be a graph, $t$ a target score, and $D$ a dominating set that we initialise to $D=\emptyset$. On alternating turns, beginning with Alice, Alice and Bob add an unchosen vertex to $D$ such that the number of dominated vertices increases. That is, the set $N[D]$ increases in size. The game stops when $D$ forms a dominating set in $G$. The score of the game, $s$, is the size of the dominating set at the end of the game. That is $s=|D|$. Alice wins if $s\leq t$ and Bob wins otherwise.

\begin{definition}[Game Domination Number]
    Let $G$ be a graph. The game dominating number $\gamma_g(G)$ is the minimum target score such that Alice always has a winning strategy.
\end{definition} 

Suppose $\CC$ is a class of graphs. $\gamma_g(\CC)$ is the smallest $k$ such that for every graph $H\in\CC$, $\gamma_g(H)\leq k$. We say a class $\CC$ is bounded above by $k$ if $\varphi_g(\CC)\leq k$. $\CC$ is bounded below by $k$ if there is a graph $H$ in $\CC$ such that $\gamma_g(H)= k$ and we write $k
leq\gamma_g(\CC)$. Hence if $k\leq\gamma_g(\CC)\leq k$ then $\gamma_g(\CC) = k$. 

%-----------------------------------------------------------------------------

\subsection{Lower Bounds for Game Dominating Number}
    
A graph $G=(E,V)$ has no dominating sets smaller than $\gamma(G)$. This means that if the target score of the dominating game is less than $\gamma(G)$ then there is no strategy that will allow Alice to win. Therefore $\gamma(G)$ is a lower bound for the game domination number. That is \[\gamma(G)\leq\gamma_g(G)\]

Theorem \ref{thm:oreDomUpper}, is a well known result in graph theory.   
\begin{theorem}[Ore 1962 \cite{oysteinore1962}] \label{thm:oreDomUpper} 
    For any connected graph $G$ with no isolated vertices and $n$ vertices,     
    \[\gamma(G) \leq \frac{n}{2}\]
\end{theorem}

In this report we will show that theorem \ref{thm:oreDomUpper} also provides us with a lower bound for the game domination number.  
Before we can show $n/2$ is a lower bound we introduce the following lemmas. 
\begin{lemma}\label{lem:ore_is_tight}
    For every $n$ there exists a connected graph $G$ with $n$ vertices and no isolated vertices such that $\gamma(G) = \left\lfloor n/2 \right\rfloor$. Hence the bound    
    $\gamma(G) \leq \frac{n}{2}$ is tight
\end{lemma}
\begin{proof}
    Fix $n$. If $n$ is odd then it suffices to show $\gamma(G) \leq (n-1)/2$. So, without loss of generality suppose $n$ is even. Consider the path graph with $n/2$ vertices with a single additional vertex attached to each vertex, denote this graph $G$. See figure \ref{fig:ext_path}.
    \begin{figure}[H]
        \centering
        \begin{tikzpicture}
        \node [normal] (v2) at (-2,0) {};
        \node [normal] (v1) at (-3,0) {};
        \node [normal] (v3) at (-1,0) {};
        \node [normal] (v4) at (0,0) {};
        \node [normal] (v5) at (1,0) {};
        \node [normal] (v10) at (-3,-1) {};
        \node [normal] (v9) at (-2,-1) {};
        \node [normal] (v8) at (-1,-1) {};
        \node [normal] (v7) at (0,-1) {};
        \node [normal] (v6) at (1,-1) {};
        \draw  (v1) edge (v2);
        \draw  (v2) edge (v3);
        \draw  (v3) edge (v4);
        \draw  (v4) edge (v5);
        \draw  (v5) edge (v6);
        \draw  (v7) edge (v4);
        \draw  (v3) edge (v8);
        \draw  (v9) edge (v2);
        \draw  (v1) edge (v10);
        \end{tikzpicture}
        \caption{The extended path graph, $G$, with 10 vertices}
        \label{fig:ext_path}
    \end{figure}
    A dominating set in $G$ is the set of vertices in the original path graph. Hence $\gamma(G)= n/2$. Therefore $n/2$ is a tight upper bound for $\gamma(\CC)$. 
\end{proof}

Lemma \ref{thm:domlow} and theorem \ref{thm:gamedomLow} are implicit in the literature. But, we state them here with proofs.

\begin{lemma}\label{thm:domlow}
    Let $\CC$ be a class of graphs and $\gamma(\CC)$ tight upper bound for the domination number of $\CC$. That is, for all $G\in\CC$, $\gamma(G)\leq\gamma(\CC)$ and there exists some $G\in\CC$ such that $\gamma(G)=\gamma(\CC)$. Then,
    \[\gamma(\CC) \leq  \gamma_g(\CC)\]
\end{lemma}
\begin{proof}
    Let $\CC$ be a class of graphs and $G\in\CC$ a graph such that $\gamma(G) = \gamma(\CC)$. 
    Therefore for $G$ we are unable to find a dominating set with less than $ \gamma(\CC) $ vertices.
    Therefore there cannot be a winning strategy for Alice with a score less than $\gamma(\CC)$.
    Therefore $\gamma(\CC) \leq \gamma_g(\CC)$
\end{proof}

\begin{theorem}\label{thm:gamedomLow}
    Let $G$ be a connected graph with $n$ vertices, such that $n \geq 4$. Then, there is a winning strategy for Alice with    
    \[  \frac{n}{2} \leq \gamma_g(G)  \]    
\end{theorem}
\begin{proof}
    Let $G$ be a connected graph with $n$ vertices. 
    By lemma \ref{lem:ore_is_tight} theorem \ref{thm:oreDomUpper} is a tight upper bound.
        
    Therefore by theorems \ref{thm:oreDomUpper} and \ref{thm:domlow}, $ n/2 \leq \gamma_g(G)$.
\end{proof}

Theorem \ref{thm:gamedomLow} does not say that for all connected graphs with domination number less than $n/2$, Alice will always lose. Rather a target score greater than $n/2$ is needed to ensure Alice will always win on any arbitrary connected graph. As an example consider a path graph $P_n$. That is the graph with $n$ vertices connected in a single line. 
\begin{figure}[h]
    \centering
    \begin{tikzpicture}
        \node [normal] (v1) at (-2.5,-0.5) {};
        \node [normal] (v2) at (-1,-0.5) {};
        \node [normal] (v6) at (5,-0.5) {};
        \node [normal] (v7) at (6.5,-0.5) {};
        \node [normal] (v8) at (8,-0.5) {};
        \node [normal] (v3) at (0.5,-0.5) {};
        \node [normal] (v4) at (2,-0.5) {};
        \node [normal] (v5) at (3.5,-0.5) {};
        
        \draw  (v1) edge (v2);
        \draw (v2) edge (v3);
        \draw (v3) edge (v4);
        \draw (v4) edge (v5);
        \draw (v5) edge (v6);
        \draw (v6) edge (v7);
        \draw (v7) edge (v8);
    \end{tikzpicture}
    \caption{The path graph $P_8$}
\end{figure}
%\todo N[v] notation
Let $D$ be the current dominating in the dominating game. In $P_n$ any vertex that Alice plays will increase the number of dominated vertices ($N[D]$) by at most 3. Bob can play a vertex in $N(D)$. Doing this will increase the size of $N[D]$ by at most 1. Hence, after each round $N(D)$ has increased by at most 4 and $D$ has increased by 2. Therefore the game will end after $n/4$ turns with $|D| = 2(n/4) = n/2$. Hence a target score of $n/2$ is needed to ensure that Alice will win.

%--------------------------------------------------------------------

\subsection{Upper bounds for Game Domination Number}
Consider some graph $G$, the vertex set $V(G)$ is a dominating set. Therefore when $D$ forms a dominating set $|D|\leq |V(G)|$. Thus a game with target score $|V(G)|$ guarantees Alice will win. Hence $|V(G)|$ is an upper bound for the game domination number. That is $\gamma_g(G) \leq |V(G)|$.

For a better upper bound we introduce a new strategy for Alice. This strategy involves Alice imagining a perfect play and using this play as a strategy. Alice imagines a minimum dominating set. This simple strategy and the following bound were first observed in Bre\v{s}ar and Klav\v{z}ar and Rall 2010 \cite{BrKlRa2010}. If Alice was playing herself this minimum set would provide a perfect score ($\gamma_g(G)=\gamma(G))$). However, Alice is not playing herself. Bob's strategy forces a less than perfect score. But, by playing her imagined strategy she will always win on a score strictly less than twice her perfect score.

\begin{theorem}[Bre\v{s}ar and Klav\v{z}ar and Rall 2010 \cite{BrKlRa2010}]\label{thm:gamedomup}
    For $G$ a graph and $\gamma(G)$ the dominating number of $G$,  
    \[\gamma_g(G)<2\gamma(G)\]
\end{theorem}
In \cite{BrKlRa2010} the authors do not give a formal proof, they give a brief sketch. This is something we aim to remedy in this report. 
\begin{proof}
    Let $G$ be a graph and $X\subseteq V(G)$ a dominating set such that $|X| = \gamma(G)$. On Alice's first turn she plays any vertex in $X$. Now, suppose Bob has just played a vertex and $D$ is the current partial dominating set. Alice's strategy is to play any unchosen vertex, $v\in X\setminus D$. After no more than $\gamma(G)$ rounds the game must have ended as $X$ is a dominating set and Alice plays a vertex from $X$ in each round. 
    
    If the game ends in the $\gamma(G)$-th round then it ended on Alice's turn. This is because after Alice's $\gamma(G)$-th turn the game is over. Thus Bob has had one less turn that Alice. In each round Alice and Bob each add one vertex to $D$. Hence the game ends with size of $D$ exactly $2\gamma(G) -1$. 
    
    If the game ends in strictly less than $\gamma(G)$ rounds then the size of $D$ is strictly less than $2\gamma(G)$.
\end{proof}
%hence bounds for game relie on bounds for dom num

If the domination number of a class of graphs is known, then we know an upper bound for the game domination number. However, the domination number of most classes of graphs is not known. This is because finding a dominating set is an NP-complete problem. And so finding dominating numbers is difficult.

Suppose the domination number is not known, but we know an upper bound for it. Then if Alice pretends that the domination number is this bound, we get an upper bound for the game domination number. Alice's strategy would be exactly the same as in theorem \ref{thm:gamedomup}. This means that we can get an improvement on the upper bound of the domination number by finding  better upper bounds for the game domination number. However, finding bounds for the dominating number is beyond the scope of this report. But, we provide some examples,

%todo add more bounds based on classes of graphs

\subsection{The Domination number in Trees}
Finding better bounds for the class of all graphs is a difficult problem. But for other, smaller, classes of graphs better bounds can be found. 
\begin{definition}[Forest]
    A graph is a \textit{forest} if it contains no cycles. 
\end{definition}
\begin{definition}[Tree]
    A graph is s \textit{tree} if it is connected and contains no cycles.
\end{definition}
For the class of trees Kinnersley, West, and Zamani 2013 \cite{KiWeZa2013} conjectured the following. 
\begin{conjecture}[Kinnersley, West, and Zamani 2013 \cite{KiWeZa2013}] \label{conj:treedomup}
    For a forest $G$ with $n$ vertices and no isolated vertices,    
    \[ \gamma_g(T) \leq \left\lceil \frac{3n}{5} \right\rceil\]
\end{conjecture}
While conjecture \ref{conj:treedomup} is still undecided it has been shown for certain types of trees. 

A \textit{caterpillar graph} is a tree in which all the vertices lie on a path or have distance at most one from a central path. See figure \ref{fig:catgraphs} for some examples.
%
\begin{figure} [h]
    \centering
    \begin{subfigure}{.3\textwidth}
        \centering
        \begin{tikzpicture}[scale=.7]
        \node [normal] (v5) at (-2.5,-2) {};
        \node [normal] (v4) at (-1.5,-1) {};
        \node [normal] (v3) at (-0.5,0) {};
        \node [normal] (v2) at (0.5,1) {};
        \node [normal] (v6) at (-2,0) {};
        \node [normal] (v8) at (0.5,-1) {};
        \node [normal] (v7) at (0,-1.5) {};
        \node [normal] (v1) at (-0.5,2.5) {};
        \draw [] (v1) edge (v2);
        \draw [] (v2) edge (v3);
        \draw [] (v3) edge (v4);
        \draw [] (v4) edge (v5);
        \draw [] (v6) edge (v4);
        \draw [] (v3) edge (v7);
        \draw [] (v8) edge (v3);
        \end{tikzpicture}
    \end{subfigure}
\begin{subfigure}{.3\textwidth}
    \centering
    \begin{tikzpicture}[scale=.7]
    \node [normal] (v1) at (0,0) {};
    \node [normal] (v2) at (1,1) {};
    \node [normal] (v3) at (2.5,1.5) {};
    \node [normal] (v4) at (4,0.5) {};
    \draw [] (v1) edge (v2);
    \draw [] (v2) edge (v3);
    \draw [] (v3) edge (v4);
    \node [normal] (v5) at (5.5,0) {};
    \draw [] (v5) edge (v4);
    \end{tikzpicture}
\end{subfigure}
\begin{subfigure}{.3\textwidth}
    \centering
    \begin{tikzpicture}[scale=.7]
    \node [normal] (v1) at (0,0) {};
    \node [normal] (v3) at (0,1) {};
    \node [normal] (v9) at (0,2) {};
    \node [normal] (v4) at (-1.5,0) {};
    \node [normal] (v2) at (-2,0.5) {};
    \node [normal] (v5) at (1.5,0) {};
    \node [normal] (v6) at (2,0.5) {};
    \draw [] (v2) edge (v3);
    \draw [] (v3) edge (v4);
    \draw [] (v3) edge (v5);
    \draw [] (v6) edge (v3);
    \draw [] (v1) edge (v3);
    \node [normal] (v10) at (1,2) {};
    \node [normal] (v11) at (1,3) {};
    \node [normal] (v12) at (1.5,3) {};
    \draw [] (v9) edge (v10);
    \draw [] (v10) edge (v11);
    \draw [] (v10) edge (v12);
    \draw [] (v9) edge (v3);
    \end{tikzpicture}
\end{subfigure}
    \caption{Some caterpillar graphs}
    \label{fig:catgraphs}
\end{figure} 
%
Kinnersley, West, and Zamani 2013 \cite{KiWeZa2013} showed conjecture \ref{conj:treedomup} holds for $n$-vertex forests with no isolated vertices where each component is a caterpillar. 

\begin{theorem}[Kinnersley, West, and Zamani 2013 \cite{KiWeZa2013}]\label{thm:domcat}
    For $F$ a forest of caterpillars with no isolated vertices 
    $\gamma_g(F)\leq 3n/5$ 
\end{theorem}
%\begin{proof}
    
Rather than provide a full proof we give the main ideas; namely we describe Alice's strategy for $F$ a forest of caterpillars with no isolated vertices. 
    %To extend the strategy to a full proof we would proceed by induction on the number of turns. Then verify that the theorem holds on induced subgraphs.
    %We show this bound by induction on $n$ the number of vertices in the forest $F$. For $n\leq 5$ the hypothesis  follows from an analysis of the few possible graphs.
    %
    %For the inductive step assume that $n\geq 6$. 
    Let $D$ be the current partial dominating set on Alice's turn. A vertex is \textit{totally dominated} if all it's neighbours are in the dominating set. The \textit{residual graph} $F'$ of $F$ is $F$ less any vertices that are totally dominated. That is $F'=F\ba\{v\in V(F): N(v)\subseteq N[D]\}$. The residual graph $F'$ is the subgraph that contains only valid moves. Alice only considers the graph $F'$ when deciding on her next play. Alice considers three cases, in order of preference.
    %
    \begin{enumerate}        
    \item $F'$ has a vertex $v$ incident to two leaves. Alice plays $v$. 
    %Let $F'$ be $F$ less any vertices that are totally dominated. That is $F'=F\ba\{v\in V(F): N(v)\subseteq D\}$. Then by the inductive hypothesis $\gamma_g(F) \leq 1 + \gamma_g'(F') < 3n/5$.
    
    \item $F'$ has a component $C$ isomorphic to one of the path graphs $P_2,P_4,P_5$. Alice plays a winning strategy on $C$. 
    
    \item $F'$ has no vertices incident to 2 leaves and no component isomorphic to one of the path graphs $P_2,P_4,P_5$. If this is the case then no component of $F'$ has less than 6 vertices. Fix a component $C$ of $F$ and let $u_1,\dots,u_k$ the longest path in $C$. If there is some $i$ such that $d(u_i)=3$ then Alice plays $u_i$. If not, then Alice plays $u_6$.    
    \end{enumerate}
Alice keeps playing according to these rules until there are no more vertices to play. This concludes the strategy.
%The remainder of the proof done using induction on the number vertices in the graph. We would use the fact that if $F$ is the union of two forests, that is $F=F_1\oplus F_2$, then $\gamma_g(F)\leq\gamma_g(F_1) +\gamma_g'(F_2)$. 
%By using this strategy, Alice will always win on $F$ if the target score is less than or equal to $3n/5$.
%    \begin{enumerate}
%        \item $\gamma_g'(F)\leq 1 +\gamma_g(F-v) = 1 + 3(n-1)/5 = (3n+2)/5$
%        \item $\gamma_g(F_1\oplus F_2) \leq \gamma_g(F_1) + \gamma_g'(F_2)$
%        \item $\gamma_g(F) \leq 1 + \gamma'(F')$.
%    \end{enumerate}
%    
%    First, $F$ has a vertex $v$ incident to two leaves. Then after Alice plays $v$, $F'$ has $n-3$ vertices. So, 
%    \[\gamma_g(F)\leq 1 + \gamma_g(F') \leq 1 + \frac{3(n-3)+2}{5} = \frac{3n -2}{5} < \frac{3n}{5}\]
%    
%\end{proof}
Further work has been done to improve the class of trees that the conjecture holds for. Bujt\'{a}s, Csilla 2015 \cite{Bujt2015} showed that conjecture \ref{conj:treedomup} holds for forests that have no isolated vertices and in which no two leaves have distance 4.
%
\begin{theorem}[Bujt\'{a}s, Csilla 2015 \cite{Bujt2015}]\label{thm:domtreedist4}
    For $F$ a forest with no isolated vertices such that no two leaves have distance 4,
    $\gamma_g(F)\leq \frac{3n}{5}$.
\end{theorem}
Again, rather than provide a full proof we give the main ideas; namely we describe Alice's strategy. As part of this strategy Alice assigns a value to each vertex $v\in V(F)$ as follows, 
\begin{itemize}
    \item $v$ has value 3 if it is not dominated.
    \item $v$ has value 2 if it is dominated but has an undominated neighbour.
    \item $v$ has value 0 if it is totally dominated.
\end{itemize}
The \textit{gain} of a play is the difference between the sum of all the values in the tree before and after each play. In Alice's strategy for theorem \ref{thm:domtreedist4} there are 4 stages. At each stage Alice's strategy changes. The game starts in stage 1. If on Alice's turn there is no suitable vertex at that stage then she moves to the next stage. 
\begin{enumerate}[St{a}ge 1:]
    \item There is a vertex $v$ with gain at least 7 such that $D\cup\{v\}$ totally dominates two new vertices. Alice plays $v$.
    \item There is a vertex $v$ with gain at least 7. Alice plays $v$.
    \item There is a vertex $v$ with gain at least 6. Alice plays a vertex $u$ with maximum gain. Subject to maximum gain, Alice plays a vertex with value 3 that is incident to a leaf with value 3.
    \item There is a vertex $v$ with gain at least 3. Alice plays $v$.
\end{enumerate}
This continues until there are no more vertices with gain at least 3. At which point every vertex is dominated as every undominated vertex has score 3. This concludes the strategy.

\section{The $(a,b)$-dominating game}

In the dinner party problem, Alice and Bob each only place one platter in an alternating sequence. But, suppose instead Alice and Bob are each allowed to place 2 platters on their turn. The question here is, does this change the minimum number of platters Alice needs? What about when Alice places more platters than Bob in each turn?

This extended dinner party problem is a natural extension of the dominating game We allow Alice and Bob to select more than one vertex on each turn. We call this extension as the $(a,b)$-dominating game. In the $(a,b)$-dominating game the only difference is Alice plays $a$ vertices per turn and Bob plays $b$ vertices.

As far as we are aware there has been no published work on the $(a,b)$-dominating game. 

\begin{definition}[$(a,b)$-Game Domination Number]
    Let $G$ be a graph and $a,b\in\Nn$ integers. The game domination number $\gamma_g(G;a,b)$ is the smallest target score such that Alice has a winning strategy when playing the $(a,b)$-dominating game.
\end{definition}

\begin{theorem}[Askes] 
    Let $G$ be a graph with $n$ vertices and $a,b\geq1$ integers. Then,
%    \[\gamma_{g}(G;1,b) \geq (b+1)\gamma(G) \]    
    \[\gamma_{g}(G;a,b) \leq \frac{a+b}{a}\gamma(G) - 1\]
\end{theorem}
    The following proof is an extension of the proof of theorem \ref{thm:gamedomup} to the $(a,b)$-dominating game. In fact when $a=b=1$ the proof is the same as the proof of theorem \ref{thm:gamedomup}.
\begin{proof}    
   Let $G$ be a graph and $X\subseteq V(G)$ a dominating set of $G$ such that $|X| = \gamma(G)$. On Alice's first move she plays any $a$ vertices in $X$. Now, suppose Bob has just played $b$ vertices and $D$ is the current partial dominating set. Alice's strategy is to play any $a$ unchosen vertices, $v_1,\dots,v_a \in X\setminus D$. After no more than $\gamma(G)/a$ rounds the game must have ended as $X$ is a dominating set and at least $a$ elements of $X$ are chosen each round. In each round Alice adds $a$ vertices to $D$ and Bob adds $b$ vertices to $D$, hence the size of $D$ is at most $(a+b)/a$ times the number of rounds. 
   
   If the number of rounds is exactly $\gamma(G)/a$ then the game ends on Alice's turn. This is because Alice starts each round and the game is over on her $\gamma(G)/a$-th turn. In such a case Bob has had one less turn than Alice. Hence $|D|=(a+b)\frac{\gamma(G)}{a} -b$. If the number of rounds is strictly less than $\gamma(G)/a$ then the size of $D$ is strictly less than $(a+b)\frac{\gamma(G)}{a}$. In either case the game ends with $|D| \leq \frac{a+b}{a}\gamma(G) - 1$.
\end{proof}

\section{The Independent Dominating Game} \label{sec:ind_dom_game}

To motivate the independent dominating game we introduce the cover band \textit{David and the Derivatives}. David and the Derivatives are the headline act at a large concert. On a normal night there would be no problems filling the seats in the concert hall. The problem is, they are under COVID restrictions. This means that all the guests must be seated 2 meters apart. Alice works for the Ministry of health. Her job is to seat as few people as possible while still filling the hall. Bob is David and the Derivatives' manager. His job to maximise the number of people seated in the hall. There are only door sales for this concert. So the more people Bob can seat the more profit he and David and the Derivatives make. The guests arrive one by one. Alternating, Alice and Bob direct people to seats. This continues until no more people can be seated without breaking social distancing. Denoting the number of people seated as $n$. How small can Alice guarantee $n$ will be? And, how large can Bob guarantee $n$ will be? These questions have non-trivial solutions. And are a specific instance of the Independent Dominating Game.

\begin{definition}[Independent Set]
    For a graph $G$, an \textit{independent Set} is a subset $I\subseteq V(G)$ such that no two vertices in $I$ are adjacent. That is, for all $u,v\in I$ there exists no edge $(u,v)\in E(G)$.
\end{definition}

\begin{definition}[Independent Dominating Set]
   For a graph $G$, an \textit{independent dominating set} is a set of vertices in $G$ that is both independent and dominating.
\end{definition}

The independent domination game is played on a graph $G$ as follows. Starting with Alice, the players take turns playing a vertex $v$ such that $v$ is not incident with any vertex that is already dominated. That is $v\in V(G)\backslash N[D]$, where $D$ is the current dominating set. Alice and Bob continue playing vertices until $D$ forms a dominating set. The score of the game is the size of the dominating set. Alice's goal is to minimise the score, and Bob's goal is to maximise it.


Since an independent dominating set is a maximal independent set and vice versa the independent dominating game is equivalent to the Independent set game. This game is called the \textit{competition-independence game}. In this game Alice and Bob take turns constructing a maximal independent set. As before Alice tries to minimise the size of the independent set. Bob tries to maximise the independent set. The \text{competition-independence number} $I_A(G)$ (respectively $I_B(G)$) is the optimal size of a maximal independent set when Alice starts (or Bob respectively). 

The competition-independence game was first introduced in Phillips and Slater 2001 \cite{PhillpsSlater2001}. In \cite{PhillpsSlater2001} the authors demonstrate some bounds for path graphs. In particular, they prove theorem \ref{thm:indDomPath}.
\begin{theorem}[Phillips and Slater 2001 \cite{PhillpsSlater2001}]\label{thm:indDomPath}
    For $P_n$ the path graph with $n$ vertices
    \begin{align*}
    I_A(P_n) &=\lfloor(3n+4)/4\rfloor\\
    I_B(P_n) &=\lfloor(3n+6)/4\rfloor       
    \end{align*} 
\end{theorem}
Further work on the bounds of competition-independence game was done by Goddard and Henning 2008 \cite{GoddardHenning2018}. In \cite{GoddardHenning2018} the authors proving a winning strategy for Alice on trees of with maximum degree less than 3.

\begin{theorem}[Goddard and Henning 2008 \cite{GoddardHenning2018}]    \label{thm:indDomtree}
For any tree $T$ with greater than or equal to 2 vertices and with maximum degree 3 
\[I_A(T)\leq 4n/7\]
\end{theorem}

We demonstrate the strategy for Alice introduced in \cite{GoddardHenning2018}'s proof of theorem \ref{thm:indDomtree}. Fix some tree $T$. Suppose it is Alice's turn. Let $I$ be the current independent set. Define $J_I$ to be the set of isolated vertices in $T\ba N[I]$. The \textit{energy} of $I$ is defined as 
    \[\varphi(I) = |I| + |J_I| + 4/7|(V(T) - N[I] - J_I)|\]
 Alice's strategy is to play the vertex  that minimises this energy. That is the $v\in V(T)$ that minimises $\varphi(I\cup\{v\})$

Compared to the domination and total domination game there has been comparatively little work on the competition-independence game. 
%In total there is probably less than a dozen papers on the topic. 
Recent work by Worawannotai, Ruksasakchai 2020 \cite{WorRuk2020} compares the competition-independence game to the dominating game. Most of the results in \cite{WorRuk2020} are a variation of the type: For a positive integer $n$, there is some graph $G$ such that
\[\gamma_g(G) - I_A(G) = n\]

%todo prove this result
%todo provide example
There is an interesting fact about the domination game. If Bob has the first move the domination number is less than or equal to $\gamma_g(G) +1$. To see this consider a graph $G$. Bob plays $v$ as his first move. Alice pretends she has the first move in the subgraph $H=G\ba N[v]$. Alice will then win on $G$ with score $\gamma_g(H)+1\leq \gamma_g(G)+1$.   
This is not true for the competition-independence game. For an example consider the star graph $S_7$ in figure \ref{fig:starS7}. 

\begin{figure}[h]
    \centering
    \begin{tikzpicture}  
        \node [normal, label=above right:$v$] (v1) at (0,0.5) {};
        \node [normal, label=above right:$u_1$] (v2) at (0,3) {};
        \node [normal, label=above right:$u_2$] (v3) at (2.35,1.5) {};
        \node [normal, label=above right:$u_3$] (v8) at (2,-1) {};
        \node [normal, label=above right:$u_4$] (v4) at (0.5,-1.95) {};
        \node [normal, label=above:$u_5$] (v5) at (-2,-1) {};
        \node [normal, label=above right:$u_6$] (v6) at (-2.5,0.5) {};
        \node [normal, label=above right:$u_7$] (v7) at (-2,2) {};
        \draw [thick] (v1) edge (v2);
        \draw [thick] (v1) edge (v3);
        \draw [thick] (v1) edge (v4);
        \draw [thick] (v5) edge (v1);
        \draw [thick] (v1) edge (v6);
        \draw [thick] (v7) edge (v1);
        \draw [thick] (v8) edge (v1);
    \end{tikzpicture}
    \caption{The star graph $S_7$}
    \label{fig:starS7}
\end{figure}

If Alice starts she will play $v$ and the game ends with $|I| = 1$. If Bob starts he will play a vertex $u_i$. Alice cannot play $v$ so must play a vertex $u_j$ such that $i\neq j$. The game will end with $|I| = 7$. Thus $I_A(S_7)=1$ and $I_B(S_7)=7$.

\begin{definition}[Independence number]
    The \textit{independence number} of a graph $G$, denoted $\alpha(G)$, is the size of the independent set in $G$.
\end{definition}

A graph has no independent sets larger than its independence number. Consequently, when playing the competition-independence game the size of the independent set formed is no larger than the  independence number. This means the competition-independence number is bounded above by the independence number. Hence, we get theorem \ref{thm:indgameup}.

\begin{theorem}\label{thm:indgameup}
    For a graph $G$, $I_A(G)\leq \alpha(G)$.    
\end{theorem}


%If 
%
%It was shown in Borg 2010 \cite{borg2010sharp} that for a graph $G$ with maximal degree $k$ and $n$ vertices 
%%
%\begin{equation}
%\label{equ3}
%\alpha(G) \leq n - \left\lceil\frac{n-1}{k}\right\rceil
%\end{equation}
%
%Further the authors show that this bound is sharp. 
%
%$I_A$ is bounded above by the independence number. As the bound in equation \ref{equ3} is sharp the independence number of the class of graphs with maximal degree $k$ is bounded below by $n - \left\lceil(n-1)/k\right\rceil$. Hence for the class of graphs $\mathcal{C}$ with maximal degree $k$ and $n$ vertices 
%
%\[\gamma_g(\mathcal{C}) = n -\left\lceil\frac{n-1}{k}\right\rceil\] 












