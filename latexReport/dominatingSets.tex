\chapter{Graph Domination}\label{chpt:domSet}

\section{Introduction}

In chapter \ref{chpt:into} we introduced the \textit{Dinner Party Problem}. To formalize the \textit{Dinner Party Problem} we introduce dominating sets. 

\begin{definition}[Dominating Set]
    Let $G$ be a graph. A Dominating set $D$ of $G$ is a subset of $V(G)$ such that every vertex in $V(G)$ is adjacent to at least one vertex in $D$.
\end{definition} 
\begin{definition}[Domination Number]
    Let $G$ be a graph. The Domination number $\gamma(G)$ of $G$ is the minimum size of a dominating set in $G$.
\end{definition}

One way of visualising dominating sets is to consider cell tower placement. Suppose we wish to build a 5g network in a city. We consider a graph of buildings in a city. Where the vertices are buildings and two vertices $A$ and $B$ are connected if a 5g tower placed on building $A$ one will provide service to building $B$. A minimum number of towers is a dominating set if this graph.

The size of a minimal dominating set is in some sense a measure of how unconnected a graph. A graph with a low domination number has high connectivity, and a graph with a high domination number has low connectivity. For example, a city that is densely packed requires less towers. Every new 5g tower reaches many buildings. Whereas a sparsely populated city will require more towers. Each tower only services a few buildings.   

There are various games involving dominating sets. One such game was introduced by  Alon, Noga and Balogh, J\'{o}zsef and Bollob\'{a}s, B\'{e}la and Szab\'{o}, Tam\'{a}s 2002 \cite{AlBABoSz2002}. In which the authors describe strategies for minimising the domination number in induced orientations. A dominating set in an oriented graph is the standard definition with the exception for two vertices $u,v$, $u$ is adjacent to  
\begin{definition}
    Let $G$ be an oriented graph.  A Dominating set $D$ of $G$ is a subset of $V(G)$ such for every vertex $v\in V(G)$ there exists a vertex $u\in D$ such that $v=u$ or the edge $(u,v)\in E(G)$. 
\end{definition}
The game that they describe is played as follows: Alice and Bob take turns assigning an orientation to edges in a graph $G$ until all the edges are oriented. Alice tries to minimise the size of a minimum dominating set in $G$. Bob tries to maximise the size of a minimal dominating set in $G$.     

The dominating game that is of interest to us was introduced by Bo\v{s}tjan Bre\v{s}ar, Sandi Klav\v{z}ar, and Douglas F. Rall 2010 \cite{BrKlRa2010}. In this game Alice and Bob take turns adding vertices to a set until it forms a dominating set. We will refer to this game as the dominating game.

In this chapter we show how the dominating game can be bounded in terms of both the number of vertices and the domination number. We also demonstrate a strategy for Alice that ensures she will win if the number of turns passes half the number of vertices.

%\begin{definition}
%    Independent set, maximum independent set, independence number $\alpha(G)$
%\end{definition}

\section{The Dominating game}

For a graph $G$ and a set $X\subseteq V(G)$ we denote $N[X]$ the set of neighbours of $X$ including $X$. That is $N[X] = \{v\in V(G) : \exists_{u\in X} (u,v)\in E(G)\}\cup X$. 

We define the dominating game as follows. Let $G$ be a graph, $t$ be a target score and initialise a dominating set $D=\emptyset$. On alternating turns, beginning with Alice, Alice and Bob add an unchosen vertex to $D$ such that the number of dominated vertices increases. That is the set $N(D)$ increases in size. The game stops when $D$ forms a dominating set in $G$. The score of the game, $s$, is the size if the dominating set at the end of the game, that is $s=|D|$. Alice wins if $s\leq t$ and Bob wins otherwise.

At any stage in the game we can consider the current dominating set along with the graph we are playing on. This pair forms a snapshot of the game detailing all the pertinent information. The size of $D$ tells you whose turn it is (Alice if even and Bob if odd). This snapshot is called a \textit{game state}. 

\begin{definition}[Game State]
    Let $G$ be a graph and $(v_n)_n$ a sequence of vertices in $V(G)$. $(v_n)_n$ is then a game state for $G$.
\end{definition}

A strategy in the dominating game is a prescribed next move for every possible game state. In other words a strategy tells a player how to play the game. We can define a strategy as a function from game states to game states. 

\begin{definition}[Domination Game Strategy]
    For $G$ a graph and $D=(v_n)_n$ a game state for $G$, 
    \begin{align*}        
        \varphi_G &\colon {V(G)}^{<\Nn} \to {V(G)}^{<\Nn} 
    \end{align*}
    is a strategy if it is defined on any game state and $\varphi_G(D)=D'$ gives a legal move in the dominating game. That is $D'=D\cup \{v\}$ such that $v\in V(G)\setminus D$ and is a game state for $G$. 
\end{definition}

A strategy for Alice is a strategy that is only defined on game states when it is Alice's turn. That is pairs of the form $(G,D)$ where $G$ is a graph and $D$ has even size. Equivalently a strategy for Bob is a strategy that is only defined on odd sized sets. 

For some fixed target score $t$ a winning strategy for Alice is a strategy for Alice that guarantees $D$ will form a dominating set of size less that $t$. Similarly, a winning strategy for Bob is a strategy for Bob that guarantees $D$ will form a dominating set of size greater than $t$.

\begin{definition}[Game Domination Number]
    Let $G$ be a graph. The game dominating number $\gamma_g(G)$ is the minimum target score such that Alice always has a winning strategy.
\end{definition} 

%\begin{lemma}
%    Let $G$ be a graph. 
%        
%    \[\gamma(G) \geq \alpha(G)\]
%\end{lemma}
%
%\begin{proof}
%    Let $X$ be a minimum dominating set in some graph $G=(V,E)$. By definition of dominating set vertex in $V$ is adjacent to at least one vertex in      
%\end{proof}
%    
%Recall that $\chi(G)$ is the chromatic number of the graph $G$.
%
%\begin{theorem} [Willis 3.1 \cite{Willis2011BoundsFT}] \label{willis3.1}
%    For any graph $G = (V,E)$ 
%
%    \[\alpha(G) \leq \frac{ \left | {V} \right |}{\chi(G)}\]
%\end{theorem}
%
%Recall that $\Delta(G)$ is the maximum degree of any vertex in $G$.
%
%\begin{theorem} [Balakrishnan 10.3.2 \cite{balakrishnan2012}] \label{balakrishnan10.3.2}
%    For any graph $G$ with $n$ vertices, 
%    
%    \[ \left\lceil {\frac{n}{1+\Delta(G)}} \right\rceil \leq \gamma(G) \leq n - \Delta(G)\]    
%\end{theorem}
    
For some graph $G$ $\gamma(G)$ is the size of the smallest domination set. Hence if the target score of the dominating game is less than $\gamma(G)$ then there is no strategy that will allow Alice to win. Therefore $\gamma(G)$ is a lower bound for the game domination number. That is $\gamma(G)\leq\gamma_g(G)$. 

Theorem \ref{thm:oreDomUpper}, is a well known result in graph theory. I will show that theorem \ref{thm:oreDomUpper} provides us with a lower bound for the game domination number.  
  
\begin{theorem}[Ore 1962 \cite{oysteinore1962}] \label{thm:oreDomUpper} 
    For any connected graph $G$ with $n$ vertices,     
    \[\gamma(G) \leq \frac{n}{2}\]
\end{theorem}

Before we can show $n/2$ is a lower bound we introduce the following lemma. 
\begin{lemma}[Askes]\label{thm:domlow}
    Let $\CC$ be a class of graphs and $\gamma(\CC)$ tight upper bound for the domination number of $\CC$. That is for all $G\in\CC$ $\gamma(G)\leq\gamma(\CC)$ and there exists some $G\in\CC$ such that $\gamma(G)=\gamma(\CC)$. Then,
    \[\gamma(\CC) \leq  \gamma_g(\CC)\]
\end{lemma}
\begin{proof}
    Let $\CC$ be a class of graphs and $G\in\CC$ a graph such that $\gamma(G) = \gamma(\CC)$. 
    Therefore for $G$ we are unable to find a dominating set with less than $ \gamma(\CC) $ vertices.
    Therefore there cannot be a winning strategy for Alice with a score less than $\gamma(\CC)$.
    Therefore $\gamma(\CC) \leq \gamma_g(\CC)$
\end{proof}

\begin{theorem}[Askes] \label{thm:gamedomLow}
    Let $G$ be a connected graph with $n$ vertices, such that $n \geq 4$. Then, there is a winning strategy for Alice with    
    \[  \frac{n}{2} \leq \gamma_g(G)  \]    
\end{theorem} %todo true?????????
\begin{proof}
    Let $G$ be a connected graph with $n$ vertices.
    
    Consider the complete graph with 2 vertices $K_2$. A dominating set in $K_2$ is a single vertex, and hence $\gamma(K_2)=n/2=1$. Therefore $n/2$ is a tight upper bound for $\gamma(\CC)$. 
    
    Therefore by theorems \ref{thm:oreDomUpper} and \ref{thm:domlow} $ n/2 \leq \gamma_g(G)$.
\end{proof}

Theorem \ref{thm:gamedomLow} does not say that for all connected graphs with domination number less than $n/2$, Alice will always lose. Rather a target score greater than $n/2$ is needed to ensure Alice will always win on any arbitrary connected graph. As an example consider a path graph $P_n$. That is the graph with $n$ vertices connected in a single line. 

\begin{figure}[H]
    \centering
    \begin{tikzpicture}
        \node [normal] (v1) at (-2.5,-0.5) {};
        \node [normal] (v2) at (-1,-0.5) {};
        \node [normal] (v6) at (5,-0.5) {};
        \node [normal] (v7) at (6.5,-0.5) {};
        \node [normal] (v8) at (8,-0.5) {};
        \node [normal] (v3) at (0.5,-0.5) {};
        \node [normal] (v4) at (2,-0.5) {};
        \node [normal] (v5) at (3.5,-0.5) {};
        
        \draw[ultra thick]  (v1) edge (v2);
        \draw[ultra thick] (v2) edge (v3);
        \draw[ultra thick] (v3) edge (v4);
        \draw[ultra thick] (v4) edge (v5);
        \draw[ultra thick] (v5) edge (v6);
        \draw[ultra thick] (v6) edge (v7);
        \draw[ultra thick] (v7) edge (v8);
    \end{tikzpicture}
    \caption{The path graph $P_8$}
\end{figure}
%\todo N[v] notation
Let $D$ be the current dominating in the dominating game. In $P_n$ any vertex that Alice chooses will increase the number of dominated vertices ($N[D]$) by at most 3. Bob can pick a vertex in $N(D)$. Doing this will increase the size of $N[D]$ by at most 1. Hence, after each round $N(D)$ has increased by at most 4 and $D$ has increased by 2. Therefore the game will end after $n/4$ turns with $|D| = 2(n/4) = n/2$. Hence a target score of $n/2$ is needed to ensure that Alice will win.

%thm \ref{thm:gamedomLow} is also proved in Alona, Baloghc, Bollobas, and Szabo 2002 \cite{AlBABoSz2002}. not true, is a diffferent game

An upper bound for the game domination number for a class of graphs is a target score that guarantees Alice has a winning strategy. Consider some graph $G$, the vertex set $V(G)$ is a dominating set. Therefore when $D$ forms a dominating set $|D|\leq |V(G)|$. Thus a game with target score $|V(G)|$ guarantees Alice will win. Hence $|V(G)|$ is an upper bound for the game domination number. That is $\gamma_g(G) \leq |V(G)|$. 

For a better upper bound we introduce the Imagination strategy. The imagination strategy was first introduced by Bartnicki, T. and Bre\v{s}ar, B. and Grytczuk, J. and Kov\v{s}e, M. and Miechowicz, Z. and Peterin, I. 2008 \cite{BaBrGrKoMiPe2008} in the context of the colouring game. The imagination strategy involves imagining a perfect play and using this play as a sragety for Alice. In the dominating game Alice imagines a minimum dominating set. If Alice was playing herself this minimum set would provide a perfect score ($\gamma_g(G)=\gamma(G))$). However, Alice is not playing herself and Bobs strategy forces a less than perfect score. But, by playing her imagined strategy she will always win on a score strictly less than twice her perfect score.  

\begin{theorem}\label{thm:gamedomup}
    For $G$ a graph and $\gamma(G)$ the dominating number of $G$,  
    \[\gamma_g(G)<2\gamma(G)\]
\end{theorem}
\begin{proof}
    Let $G$ be a graph and $X\subseteq V(G)$ a dominating set such that $|X| = \gamma(G)$. On Alice's first turn she picks any vertex in $X$. Suppose Bob has just chosen a vertex and $D$ is the current dominating set. Alice's strategy is too pick any unchosen vertex, $v\in X$. After $\gamma(G)$ turns the game must have ended as $X$ is a dominating set and Alice picks a vertex from $X$ in each turn. On each turn Alice and Bob each add one vertex to $D$, hence the size of $D$ is at most $2\gamma$.
\end{proof}
%hence bounds for game relie on bounds for dom num

If the domination number of a class of graphs is known then we know an upper bound for the game domination number. However, the domination number of most classes of graphs is not known. If the domination number is not known, but we know an upper bound for it. Then we have an upper bound for the game domination number. Thus, an improvement on the upper bound of the domination number gives us a better upper bound for the game domination number.
%We can see from theorem \ref{thm:gamedomup} that finding better upper bounds for the game dominating number is a matter of finding better bounds for the dominating number and is out of the scope of this paper. 

%todo add more bounds based on classes of graphs


Kinnersley, William B. and West, Douglas B. and Zamani, Reza 2013 \cite{KiWeZa2013} conjectured the following. 
\begin{conjecture}[ Kinnersley, William B. and West, Douglas B. and Zamani, Reza 2013 \cite{KiWeZa2013}] \label{thm:treedomup}
    Let $T$ be a tree with $n$ vertices. Then,    
    \[ \gamma_g(T) \leq \left\lceil \frac{3n}{5} \right\rceil\]
\end{conjecture}
 It was shown in Bujt\'{a}s, Csilla 2015 \cite{Bujt2015} that Conjecture \ref{thm:gamedomup} holds for forests with no isolated vertices and in which no two leaves have distance 4.
%\begin{proof}
%    Let $T$ be a tree with $n$ vertices. On Alice's first move she picks the vertex, $v$, with maximal degree. Initialise $A=N[v]$ to denote the activated vertices. 
%
%    Suppose Bob has just picked a vertex $u$ and let $D$ be the current dominating set in $T$. The winning strategy for Alice is the greedy strategy as follows.    
%    
%    Let $U=V(T)\ba D$ the unchosen vertices in $T$. Alice uses algorithm \ref{alg:greedy} to update $A$ and choose the next vertex.
%        
%    \begin{algorithm}[h]
%        \caption{Greedy strategy}
%        \label{alg:greedy}
%        \begin{algorithmic}[1] 
%            \State $A\gets A \cup N(u)$
%            \State let $x\in U$ be the vertex that maximises $|N(x)\ba A|$
%            \State $A\gets A \cup N(x)$            
%            \State Choose $x$            
%        \end{algorithmic}
%    \end{algorithm}
%    
%   worst case path graph requires twice the minimum of the path graph???
%   
%   with no opponent this will give n/3 thus at worst with the opponent it will take 2n/3 
%       
%    At worst Alice will add two vertices to 
%    %TODO finish this proof
%\end{proof}

\section{The $(a,b)$--dominating game}

The natural extension of the dominating game is to allow Alice and Bob to select more than one vertex on each turn. We define this extension as the $(a,b)$--dominating game when Alice is allowed to select $a$ vertices per turn and Bob $b$ vertices.

\begin{definition}[Game Domination Number]
    Let $G$ be a graph and $a,b\in\mathbb{N}$ integers. The game dominating number $\gamma_g(G;a,b)$ is the smallest target score such that Alice has a winning strategy when playing the $(a,b)$--dominating game.
\end{definition}

\begin{theorem} 
    Let $G$ be a graph with $n$ vertices and $a,b>1$ integers. Then,
%    \[\gamma_{g}(G;1,b) \geq (b+1)\gamma(G) \]    
    \[\gamma_{g}(G;a,b) \leq \frac{a+b}{a}\gamma(G)\]
\end{theorem}
    This proof proceeds in much the same way as the proof of theorem \ref{thm:gamedomup}.
\begin{proof}    
   Let $G$ be a graph and $X\subseteq V(G)$ a dominating set of $G$ such that $|X| = \gamma(G)$. One Alice's first move she chooses any $a$ vertices in $X$. Suppose Bob has just chosen $b$ vertices and $D$ is the current dominating set. Alice's strategy is to pick any $a$ unchosen vertices, $v_1,\dots,v_a \in X$. After $\frac{\gamma(G)}{a}$ rounds the game must have ended as $X$ is a dominating set and at least $a$ elements of $X$ are chosen each round. In each round Alice adds $a$ vertices to $D$ and Bob adds $b$ vertices to $D$, hence the size of $D$ is at most $(a+b)\frac{\gamma(G)}{a}$.
\end{proof}













