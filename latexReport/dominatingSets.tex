\chapter{Graph Domination}\label{chpt:domSet}

\section{Introduction}

In chapter \ref{chpt:into} we introduced the Dinner Party Problem. To formalize the Dinner Party Problem we introduce dominating sets. 

\begin{definition}[Dominating Set]
    Let $G$ be a graph. A Dominating set $D$ of $G$ is a subset of $V(G)$ such that every vertex in $V(G)$ is adjacent to at least one vertex in $D$.
\end{definition} 
\begin{definition}[Domination Number]
    Let $G$ be a graph. The Domination number $\gamma(G)$ of $G$ is the minimum size of a dominating set in $G$.
\end{definition}

One way of visualising dominating sets is to consider cell tower placement. Suppose we wish to build a 5g network in a city. We consider a graph of buildings in a city. Where the vertices are buildings and two vertices $A$ and $B$ are connected if a 5g tower placed on building $A$ one will provide service to building $B$. A minimum number of towers is a dominating set if this graph.

The size of a minimal dominating set is in some sense a measure of how unconnected a graph. A graph with a low domination number has high connectivity, and a graph with a high domination number has low connectivity. For example, a city that is densely packed requires less towers. Every new 5g tower reaching many buildings. Whereas a sparsely populated city will require more towers. Each tower only servicing only a few buildings.   

There are various games involving dominating sets. One such game was introduced by  Alon, Noga and Balogh, J\'{o}zsef and Bollob\'{a}s, B\'{e}la and Szab\'{o}, Tam\'{a}s 2002 \cite{AlBABoSz2002}. In which the authors describe strategies for minimising the domination number in induced orientations. A dominating set in an oriented graph is the standard definition with the exception for two vertices $u$ and $v$, $u$ is is dominated by $v$ if there is an edge $(u,v)$ in $E(G)$.
\begin{definition}
    Let $G$ be an oriented graph.  A Dominating set $D$ of $G$ is a subset of $V(G)$ such for every vertex $v\in V(G)$ there exists a vertex $u\in D$ such that $v=u$ or the edge $(u,v)\in E(G)$. 
\end{definition}
The game that they describe is played as follows: Alice and Bob take turns assigning an orientation to edges in a graph $G$ until all the edges are oriented. Alice tries to minimise the size of a minimum dominating set in $G$. Bob tries to maximise the size of a minimal dominating set in $G$.     

The dominating game that is of interest to us was introduced by Bo\v{s}tjan Bre\v{s}ar, Sandi Klav\v{z}ar, and Douglas F. Rall 2010 \cite{BrKlRa2010}. In this game Alice and Bob take turns adding vertices to a set until it forms a dominating set. We will refer to this game as the dominating game.

In this chapter we show how the dominating game can be bounded in terms of both the number of vertices and the domination number. We also demonstrate a strategy for Alice that ensures she will win if the number of turns passes half the number of vertices.

%\begin{definition}
%    Independent set, maximum independent set, independence number $\alpha(G)$
%\end{definition}

\section{The Dominating game}

For a graph $G$ and a set $X\subseteq V(G)$ we denote $N[X]$ the set of neighbours of $X$ including $X$. That is $N[X] = \{v\in V(G) : \exists_{u\in X} (u,v)\in E(G)\}\cup X$. 

We define the dominating game as follows. Let $G$ be a graph, $t$ be a target score and initialise a dominating set $D=\emptyset$. On alternating turns, beginning with Alice, Alice and Bob add an unchosen vertex to $D$ such that the number of dominated vertices increases. That is the set $N[D]$ increases in size. The game stops when $D$ forms a dominating set in $G$. The score of the game, $s$, is the size if the dominating set at the end of the game, that is $s=|D|$. Alice wins if $s\leq t$ and Bob wins otherwise.

At any stage in the game we can consider the current dominating set along with the graph we are playing on. This pair forms a snapshot of the game detailing all the pertinent information. The size of $D$ tells you whose turn it is (Alice if even and Bob if odd). This snapshot is called a \textit{game state}. 

\begin{definition}[Game State]
    Let $G$ be a graph and $(v_n)$ a sequence of vertices in $V(G)$. $(v_n)$ is then a \textit{game state} for $G$.
\end{definition}

A strategy in the dominating game is a prescribed next move for every possible game state. In other words a strategy tells a player how to play the game. We can define a strategy as a function from game states to game states. 

\begin{definition}[Domination Game Strategy]
    For $G$ a graph and $D=(v_n)_n$ a game state for $G$ let
    \begin{align*}        
        \varphi_G &\colon {V(G)}^{<\Nn} \to {V(G)}^{<\Nn} 
    \end{align*}
    where ${V(G)}^{<\Nn}$ denotes the set of all sequences of vertices in $V(G)$. $\varphi_G$ is a strategy if it is defined on any game state and $\varphi_G(D)=D'$ gives a legal move in the dominating game. That is $D'=D\cup \{v\}$ for some $v\in V(G)$ such that $D'$ dominates strictly more vertices than $D$. 
\end{definition}

A strategy for Alice is a strategy that is only defined on game states when it is Alice's turn. That is pairs of the form $(G,D)$ where $G$ is a graph and $D$ has even size. Equivalently a strategy for Bob is a strategy that is only defined on odd sized sets. 

For some fixed target score $t$ a winning strategy for Alice is a strategy for Alice that guarantees $D$ will form a dominating set of size less that $t$. Similarly, a winning strategy for Bob is a strategy for Bob that guarantees $D$ will form a dominating set of size greater than $t$.

The formalisation of strategies provides good background to the domination game. However to simplify proofs and intuition we will not explicitly define strategies as functions. Rather we will do it implicitly.    

\begin{definition}[Game Domination Number]
    Let $G$ be a graph. The game dominating number $\gamma_g(G)$ is the minimum target score such that Alice always has a winning strategy.
\end{definition} 

%\begin{lemma}
%    Let $G$ be a graph. 
%        
%    \[\gamma(G) \geq \alpha(G)\]
%\end{lemma}
%
%\begin{proof}
%    Let $X$ be a minimum dominating set in some graph $G=(V,E)$. By definition of dominating set vertex in $V$ is adjacent to at least one vertex in      
%\end{proof}
%    
%Recall that $\chi(G)$ is the chromatic number of the graph $G$.
%
%\begin{theorem} [Willis 3.1 \cite{Willis2011BoundsFT}] \label{willis3.1}
%    For any graph $G = (V,E)$ 
%
%    \[\alpha(G) \leq \frac{ \left | {V} \right |}{\chi(G)}\]
%\end{theorem}
%
%Recall that $\Delta(G)$ is the maximum degree of any vertex in $G$.
%
%\begin{theorem} [Balakrishnan 10.3.2 \cite{balakrishnan2012}] \label{balakrishnan10.3.2}
%    For any graph $G$ with $n$ vertices, 
%    
%    \[ \left\lceil {\frac{n}{1+\Delta(G)}} \right\rceil \leq \gamma(G) \leq n - \Delta(G)\]    
%\end{theorem}
    
For some graph $G$ $\gamma(G)$ is the size of the smallest domination set. Hence if the target score of the dominating game is less than $\gamma(G)$ then there is no strategy that will allow Alice to win. Therefore $\gamma(G)$ is a lower bound for the game domination number. That is $\gamma(G)\leq\gamma_g(G)$. 

Theorem \ref{thm:oreDomUpper}, is a well known result in graph theory. I will show that theorem \ref{thm:oreDomUpper} provides us with a lower bound for the game domination number.  
  
\begin{theorem}[Ore 1962 \cite{oysteinore1962}] \label{thm:oreDomUpper} 
    For any connected graph $G$ with no isolated vertices and $n$ vertices,     
    \[\gamma(G) \leq \frac{n}{2}\]
\end{theorem}

Before we can show $n/2$ is a lower bound we introduce the following lemma. 
\begin{lemma}[Askes]\label{thm:domlow}
    Let $\CC$ be a class of graphs and $\gamma(\CC)$ tight upper bound for the domination number of $\CC$. That is for all $G\in\CC$ $\gamma(G)\leq\gamma(\CC)$ and there exists some $G\in\CC$ such that $\gamma(G)=\gamma(\CC)$. Then,
    \[\gamma(\CC) \leq  \gamma_g(\CC)\]
\end{lemma}
\begin{proof}
    Let $\CC$ be a class of graphs and $G\in\CC$ a graph such that $\gamma(G) = \gamma(\CC)$. 
    Therefore for $G$ we are unable to find a dominating set with less than $ \gamma(\CC) $ vertices.
    Therefore there cannot be a winning strategy for Alice with a score less than $\gamma(\CC)$.
    Therefore $\gamma(\CC) \leq \gamma_g(\CC)$
\end{proof}

\begin{theorem}[Askes] \label{thm:gamedomLow}
    Let $G$ be a connected graph with $n$ vertices, such that $n \geq 4$. Then, there is a winning strategy for Alice with    
    \[  \frac{n}{2} \leq \gamma_g(G)  \]    
\end{theorem} %todo true?????????
\begin{proof}
    Let $G$ be a connected graph with $n$ vertices.
    
    Consider the complete graph with 2 vertices $K_2$. A dominating set in $K_2$ is a single vertex, and hence $\gamma(K_2)=n/2=1$. Therefore $n/2$ is a tight upper bound for $\gamma(\CC)$. 
    
    Therefore by theorems \ref{thm:oreDomUpper} and \ref{thm:domlow} $ n/2 \leq \gamma_g(G)$.
\end{proof}

Theorem \ref{thm:gamedomLow} does not say that for all connected graphs with domination number less than $n/2$, Alice will always lose. Rather a target score greater than $n/2$ is needed to ensure Alice will always win on any arbitrary connected graph. As an example consider a path graph $P_n$. That is the graph with $n$ vertices connected in a single line. 

\begin{figure}[H]
    \centering
    \begin{tikzpicture}
        \node [normal] (v1) at (-2.5,-0.5) {};
        \node [normal] (v2) at (-1,-0.5) {};
        \node [normal] (v6) at (5,-0.5) {};
        \node [normal] (v7) at (6.5,-0.5) {};
        \node [normal] (v8) at (8,-0.5) {};
        \node [normal] (v3) at (0.5,-0.5) {};
        \node [normal] (v4) at (2,-0.5) {};
        \node [normal] (v5) at (3.5,-0.5) {};
        
        \draw[ultra thick]  (v1) edge (v2);
        \draw[ultra thick] (v2) edge (v3);
        \draw[ultra thick] (v3) edge (v4);
        \draw[ultra thick] (v4) edge (v5);
        \draw[ultra thick] (v5) edge (v6);
        \draw[ultra thick] (v6) edge (v7);
        \draw[ultra thick] (v7) edge (v8);
    \end{tikzpicture}
    \caption{The path graph $P_8$}
\end{figure}
%\todo N[v] notation
Let $D$ be the current dominating in the dominating game. In $P_n$ any vertex that Alice plays will increase the number of dominated vertices ($N[D]$) by at most 3. Bob can play a vertex in $N(D)$. Doing this will increase the size of $N[D]$ by at most 1. Hence, after each round $N(D)$ has increased by at most 4 and $D$ has increased by 2. Therefore the game will end after $n/4$ turns with $|D| = 2(n/4) = n/2$. Hence a target score of $n/2$ is needed to ensure that Alice will win.

%thm \ref{thm:gamedomLow} is also proved in Alona, Baloghc, Bollobas, and Szabo 2002 \cite{AlBABoSz2002}. not true, is a diffferent game

Suppose Alice has a winning strategy on the class of graphs $\CC$ with target score $t$. $t$ is then an upper bound for the game domination number for $\CC$. Consider some graph $G$, the vertex set $V(G)$ is a dominating set. Therefore when $D$ forms a dominating set $|D|\leq |V(G)|$. Thus a game with target score $|V(G)|$ guarantees Alice will win. Hence $|V(G)|$ is an upper bound for the game domination number. That is $\gamma_g(G) \leq |V(G)|$. 

%For a better upper bound we introduce the Imagination strategy. The imagination strategy was first introduced by Bartnicki, T. and Bre\v{s}ar, B. and Grytczuk, J. and Kov\v{s}e, M. and Miechowicz, Z. and Peterin, I. 2008 \cite{BaBrGrKoMiPe2008} in the context of the colouring game. 
For a better upper bound we introduce a new strategy for Alice. This strategy involves Alice imagining a perfect play and using this play as a strategy. Alice imagines a minimum dominating set. This simple strategy and the following bound were first observed in Bre\v{s}ar and Klav\v{z}ar and Rall 2010 \cite{BrKlRa2010}. If Alice was playing herself this minimum set would provide a perfect score ($\gamma_g(G)=\gamma(G))$). However, Alice is not playing herself. Bobs strategy forces a less than perfect score. But, by playing her imagined strategy she will always win on a score strictly less than twice her perfect score.

\begin{theorem}[Bre\v{s}ar and Klav\v{z}ar and Rall 2010 \cite{BrKlRa2010}]\label{thm:gamedomup}
    For $G$ a graph and $\gamma(G)$ the dominating number of $G$,  
    \[\gamma_g(G)<2\gamma(G)\]
\end{theorem}
In \cite{BrKlRa2010} the authors do not give a formal proof. Rather they give the inspiration for the proof. Here we aim to formalise a proof.
\begin{proof}
    Let $G$ be a graph and $X\subseteq V(G)$ a dominating set such that $|X| = \gamma(G)$. On Alice's first turn she plays any vertex in $X$. Now, suppose Bob has just played a vertex and $D$ is the current dominating set. Alice's strategy is to play any unchosen vertex, $v\in X\setminus D$. After no more than $\gamma(G)$ rounds the game must have ended as $X$ is a dominating set and Alice played a vertex from $X$ in each round. 
    
    If the game ends in exactly $\gamma(G)$ rounds then it ended on Alice's turn. In this case Bob has had one less turn that Alice. In each round Alice and Bob each add one vertex to $D$. Hence the game ends with size of $D$ exactly $2\gamma(G) -1$. 
    
    If the game ends in strictly less than $\gamma(G)$ rounds then the size of $D$ is strictly less than $2\gamma(G)$.
\end{proof}
%hence bounds for game relie on bounds for dom num

If the domination number of a class of graphs is known then we know an upper bound for the game domination number. However, the domination number of most classes of graphs is not known. If the domination number is not known, but we know an upper bound for it. Then we have an upper bound for the game domination number. Thus, an improvement on the upper bound of the domination number gives us a better upper bound for the game domination number.
%We can see from theorem \ref{thm:gamedomup} that finding better upper bounds for the game dominating number is a matter of finding better bounds for the dominating number and is out of the scope of this paper. 

%todo add more bounds based on classes of graphs

Finding better bounds for the class of all graphs is a difficult problem. But for other, smaller, classes of graphs better bounds can be found. For example the class of tress
Kinnersley, William B. and West, Douglas B. and Zamani, Reza 2013 \cite{KiWeZa2013} conjectured the following. 
\begin{conjecture}[Kinnersley, William B. and West, Douglas B. and Zamani, Reza 2013 \cite{KiWeZa2013}] \label{conj:treedomup}
    For a forest $G$ with $n$ vertices and no isolated vertices,    
    \[ \gamma_g(T) \leq \left\lceil \frac{3n}{5} \right\rceil\]
\end{conjecture}
While conjecture \ref{conj:treedomup} is still undecided it has been shown for certain types of trees. 

A \textit{caterpillar graph} is a tree in which all the vertices lie on a path or have distance at most one from a central path. See figure \ref{fig:catgraphs} for some examples.
%
\begin{figure} [h]
    \centering
    \begin{subfigure}{.3\textwidth}
        \centering
        \begin{tikzpicture}[scale=.7]
        \node [normal] (v5) at (-2.5,-2) {};
        \node [normal] (v4) at (-1.5,-1) {};
        \node [normal] (v3) at (-0.5,0) {};
        \node [normal] (v2) at (0.5,1) {};
        \node [normal] (v6) at (-2,0) {};
        \node [normal] (v8) at (0.5,-1) {};
        \node [normal] (v7) at (0,-1.5) {};
        \node [normal] (v1) at (-0.5,2.5) {};
        \draw [ultra thick] (v1) edge (v2);
        \draw [ultra thick] (v2) edge (v3);
        \draw [ultra thick] (v3) edge (v4);
        \draw [ultra thick] (v4) edge (v5);
        \draw [ultra thick] (v6) edge (v4);
        \draw [ultra thick] (v3) edge (v7);
        \draw [ultra thick] (v8) edge (v3);
        \end{tikzpicture}
    \end{subfigure}
\begin{subfigure}{.3\textwidth}
    \centering
    \begin{tikzpicture}[scale=.7]
    \node [normal] (v1) at (0,0) {};
    \node [normal] (v2) at (1,1) {};
    \node [normal] (v3) at (2.5,1.5) {};
    \node [normal] (v4) at (4,0.5) {};
    \draw [ultra thick] (v1) edge (v2);
    \draw [ultra thick] (v2) edge (v3);
    \draw [ultra thick] (v3) edge (v4);
    \node [normal] (v5) at (5.5,0) {};
    \draw [ultra thick] (v5) edge (v4);
    \end{tikzpicture}
\end{subfigure}
\begin{subfigure}{.3\textwidth}
    \centering
    \begin{tikzpicture}[scale=.7]
    \node [normal] (v1) at (0,0) {};
    \node [normal] (v3) at (0,1) {};
    \node [normal] (v9) at (0,2) {};
    \node [normal] (v4) at (-1.5,0) {};
    \node [normal] (v2) at (-2,0.5) {};
    \node [normal] (v5) at (1.5,0) {};
    \node [normal] (v6) at (2,0.5) {};
    \draw [ultra thick] (v2) edge (v3);
    \draw [ultra thick] (v3) edge (v4);
    \draw [ultra thick] (v3) edge (v5);
    \draw [ultra thick] (v6) edge (v3);
    \draw [ultra thick] (v1) edge (v3);
    \node [normal] (v10) at (1,2) {};
    \node [normal] (v11) at (1,3) {};
    \node [normal] (v12) at (1.5,3) {};
    \draw [ultra thick] (v9) edge (v10);
    \draw [ultra thick] (v10) edge (v11);
    \draw [ultra thick] (v10) edge (v12);
    \draw [ultra thick] (v9) edge (v3);
    \end{tikzpicture}
\end{subfigure}
    \caption{Some caterpillar graphs}
    \label{fig:catgraphs}
\end{figure} 
%
In \cite{KiWeZa2013} the authors proved conjecture \ref{conj:treedomup} for $n$--vertex forests with no isolated vertices where each component is a caterpillar. 

\begin{theorem}[Kinnersley, William B. and West, Douglas B. and Zamani, Reza 2013 \cite{KiWeZa2013}]\label{thm:domcat}
    For $F$ a forest of caterpillars with no isolated vertices then 
    $\gamma_g(F)\leq 3n/5$ 
\end{theorem}
Rather than provide a full proof we describe Alice's strategy for $F$ a forest of caterpillars with no isolated vertices. 
    %We show this bound by induction on $n$ the number of vertices in the forest $F$. For $n\leq 5$ the hypothesis  follows from an analysis of the few possible graphs.
    %
    %For the inductive step assume that $n\geq 6$. 
    Let $D$ be the current partial dominating set on Alice's turn. The \textit{residual graph} $F'$ of $F$ is $F$ less any vertices that are totally dominated. That is $F'=F\ba\{v\in V(F): N(v)\subseteq N[D]\}$. In the residual graph $F'$ is the subgraph that contains only valid moves. Alice only considers the graph $F'$ when deciding on her next play. Alice considers three cases.
    %
    \begin{enumerate}        
    \item $F'$ has a vertex $v$ incident to two leaves. Alice plays $v$. 
    %Let $F'$ be $F$ less any vertices that are totally dominated. That is $F'=F\ba\{v\in V(F): N(v)\subseteq D\}$. Then by the inductive hypothesis $\gamma_g(F) \leq 1 + \gamma_g'(F') < 3n/5$.
    
    \item $F'$ has a component $C$ isomorphic to one of the path graphs $P_2,P_4,P_5$. Alice plays a winning strategy on $C$. 
    
    \item $F'$ has no vertices incident to 2 leaves and no component isomorphic to one of the path graphs $P_2,P_4,P_5$. If this is the case then no component of $F'$ has less than 6 vertices. Fix a component $C$ of $F$ and let $u_1,\dots,u_k$ the longest path in $C$. If there is some $i$ such that $d(u_i)=3$ then Alice plays $u_i$. If not, then Alice plays $u_6$.    
    \end{enumerate}
By using this strategy Alice will always win on $F$ if the target score is less than $3n/5$.
    
Further work has been done to improve the class of trees that the conjecture holds for. In Bujt\'{a}s, Csilla 2015 \cite{Bujt2015} it was shown that conjecture \ref{conj:treedomup} holds for forests that have no isolated vertices and in which no two leaves have distance 4.
%
\begin{theorem}[Bujt\'{a}s, Csilla 2015 \cite{Bujt2015}]\label{thm:domtreedist4}
    For $T$ a forest with no isolated vertices such that no two leaves have distance 4
    $\gamma_g(T)\leq 3n/5$.
\end{theorem}
To decide which vertex to play Alice assigns a value to each vertex $v$ as follows: 
\begin{itemize}
    \item $v$ has value 3 if it is not dominated.
    \item $v$ has value 2 is it is dominated but has an undominated neighbour.
    \item $v$ has value 0 if it is totally dominated.
\end{itemize}
The gain of a play is the difference between the sum of all the values in the tree before and after each play. In Alice's strategy for theorem \ref{thm:domtreedist4} there are 4 stages. At each stage Alice's strategy changes. The game starts in stage 1. If on Alice's turn there is no suitable vertex at that stage then she moves to the next stage. 
\begin{enumerate}[St{a}ge 1:]
    \item There is a vertex $v$ with gain greater than 7 such that $D\cup\{v\}$ totally dominates two new vertices. Alice plays $v$.
    \item There is a vertex $v$ with gain greater than 7. Alice plays $v$.
    \item There is a vertex $v$ with gain greater than 6. Alice plays a vertex $u$ with maximum gain. Subject to maximum gain Alice plays a vertex with value 3 that is incident to a leaf with value 3.
    \item There is a vertex $v$ with gain greater than 3. Alice plays $v$.
\end{enumerate}

\section{The $(a,b)$--dominating game}

In the dinner party problem Alice and Bob each only place one platter in an alternating sequence. But, suppose instead Alice and Bob are each allowed to place 2 platters on their turn. The question here is does this change the minimum number of platters Alice needs? What about when Alice places more platters than Bob in each turn.

This extended dinner party problem is a natural extension of the dominating game We allow Alice and Bob to select more than one vertex on each turn. We call this extension as the $(a,b)$--dominating game. In the $(a,b)$--dominating game the only difference is Alice plays $a$ vertices per turn and Bob plays $b$ vertices.

\begin{definition}[$(a,b)$--Game Domination Number]
    Let $G$ be a graph and $a,b\in\Nn$ integers. The game domination number $\gamma_g(G;a,b)$ is the smallest target score such that Alice has a winning strategy when playing the $(a,b)$--dominating game.
\end{definition}

\begin{theorem}[Askes] 
    Let $G$ be a graph with $n$ vertices and $a,b>1$ integers. Then,
%    \[\gamma_{g}(G;1,b) \geq (b+1)\gamma(G) \]    
    \[\gamma_{g}(G;a,b) \leq \frac{a+b}{a}\gamma(G) - 1\]
\end{theorem}
    The following proof is an extension of the proof of theorem \ref{thm:gamedomup} to the $(a,b)$--dominating game. In fact when $a=b=1$ the proof is the same as the proof of theorem \ref{thm:gamedomup}.
\begin{proof}    
   Let $G$ be a graph and $X\subseteq V(G)$ a dominating set of $G$ such that $|X| = \gamma(G)$. On Alice's first move she plays any $a$ vertices in $X$. Now, suppose Bob has just played $b$ vertices and $D$ is the current partial dominating set. Alice's strategy is to play any $a$ unchosen vertices, $v_1,\dots,v_a \in X\setminus D$. After no more than $\gamma(G)/a$ rounds the game must have ended as $X$ is a dominating set and at least $a$ elements of $X$ are chosen each round. In each round Alice adds $a$ vertices to $D$ and Bob adds $b$ vertices to $D$, hence the size of $D$ is at most $(a+b)/a$ times the number of rounds. If the number of rounds is exactly $\gamma(G)/a$ then the game ends on Alice's turn. In such a case Bob has had one less turn than Alice. Hence $|D|=(a+b)\frac{\gamma(G)}{a} -b$. If the number of rounds is strictly less than $\gamma(G)/a$ then the size of $D$ is strictly less than $(a+b)\frac{\gamma(G)}{a}$. In either case the game ends with $|D| \leq \frac{a+b}{a}\gamma(G) - 1$.
\end{proof}

\section{The Independent Dominating Game}

To inspire the independent dominating game we introduce the cover band \textit{David and the Derivatives}. David and the Derivatives are the headline act at a large concert. On a normal night there would be no problems filling the seats in the concert hall. The problem is, they are under COVID restrictions. This means that all the guests must be seated 2 meters apart. Alice works for the Ministry of health. Her job is to seat as few people as possible while still filling the hall. Bob is David and the Derivatives' manager. His job to maximise the number of people seated in the hall. There are only door sales for this concert. So the more people Bob can seat the more profit he and David and the Derivatives make. The guests arrive one by one. Alternating, Alice and Bob direct people to seats. This continuous until no more people can be seated without breaking social distancing. Denoting the number of people seated as $n$. How small can Alice guarantee $n$ will be? And, how large can Bob guarantee $n$ will be? These questions have non-trivial solutions. And are a specific instance of the Independent Dominating Game.

\begin{definition}[Independent Set]
    For a graph $G$, an \textit{independent Set} is a subset $I\subseteq V(G)$ such that no two vertices in $I$ are adjacent. That is, for all $u,v\in I$ there exists no edge $(u,v)\in E(G)$.
\end{definition}

\begin{definition}[Independent Dominating Set]
   For a graph $G$, an \textit{independent dominating set} is a set of vertices in $G$ that is both independent and dominating.
\end{definition}

The independent domination game is played on a graph $G$ as follows. Starting with Alice. The players take turns playing a vertex $v$ such that $v$ is not incident with any vertex that is already dominated. That is $v\in V(G)\backslash N[D]$, where $D$ is the current dominating set. Alice and Bob continue playing vertices until $D$ forms a dominating set. The score of the game is the size of the dominating set. Alice's goal is to minimise the score, and Bob's goal is to maximise it.


Since an independent dominating set is a maximal independent set and vice versa the independent dominating game is equivalent to the Independent set game. This game is called the \textit{competition-independence game}. In this game Alice and Bob take turns constructing a maximal independent set. As before Alice tries to minimise the size of the independent set. Bob tries to maximise the independent set. The \text{competition-independence number} $I_A(G)$ (respectively $I_B(G)$) is the optimal size of a maximal independent set when Alice starts (or Bob respectively). 

The competition-independence game was first introduced in Phillips and Slater 2001 \cite{PhillpsSlater2001}. In \cite{PhillpsSlater2001} the authors demonstrate some bounds for path graphs. In particular, they prove theorem \ref{thm:indDomPath}.
\begin{theorem}[Phillips and Slater 2001 \cite{PhillpsSlater2001}]\label{thm:indDomPath}
    For $P_n$ the path graph with $n$ vertices
    \begin{align*}
    I_A(P_n) &=\lfloor(3n+4)/4\rfloor\\
    I_B(P_n) &=\lfloor(3n+6)/4\rfloor       
    \end{align*} 
\end{theorem}
Further work on bounds competition-independence game was done by Goddard and Henning 2008 \cite{GoddardHenning2018}. In \cite{GoddardHenning2018} the authors proving a winning strategy for Alice on trees of with maximum degree less than 3.

\begin{theorem}[Goddard and Henning 2008 \cite{GoddardHenning2018}]    \label{thm:indDomtree}
For any tree $T$ with greater than or equal to 2 vertices and with maximum degree 3 
\[I_A(T)\leq 4n/7\]
\end{theorem}

We demonstrate the strategy for Alice introduced in theorem \ref{thm:indDomtree}. Fix some tree $T$. Suppose it is Alice's turn. Let $I$ be the current independent set. Define $J_I$ to be the set of isolated vertices in $T\ba N[I]$. The \textit{energy} of $I$ is defined as \[\varphi(I) = |I| + |J_I| + 4/7|V(T) - N[I] - J_I|\]. Alice's strategy is to play the vertex  that minimises this energy. That is the $v\in V(T)$ that minimises $\varphi(I\cup\{v\})$

Compared to the domination and total domination game there has been comparatively little work on the competition-independence game. 
%In total there is probably less than a dozen papers on the topic. 
Recent work has been done in Worawannotai, Ruksasakchai 2020 \cite{WorRuk2020} comparing the competition-independence game to the dominating game. Most of the results in \cite{WorRuk2020} are a variation of the form: For a positive integer $n$, there is some graph $G$ such that
\[\gamma_g(G) - I_A(G) = n\]

%todo prove this result
%todo provide example
There is an interesting fact about the domination game. If Bob has the first move the Bob start domination number is less than or equal to $\gamma_g(G) +1$. To see this consider a graph $G$. Bob plays $v$ as his first move. Alice pretends he has the first move the subgraph $G'=G\ba N[v]$. Alice will then win on $G$ with score $\gamma_g(G')+1\leq \gamma_g(G)+1$.   
This is not true for the competition-independence game. For an example consider the star graph $S_7$ in figure \ref{fig:starS7}. 

\begin{figure}[h]
    \centering
    \begin{tikzpicture}  
        \node [normal, label=above right:$v$] (v1) at (0,0.5) {};
        \node [normal, label=above right:$u_1$] (v2) at (0,3) {};
        \node [normal, label=above right:$u_2$] (v3) at (2.35,1.5) {};
        \node [normal, label=above right:$u_3$] (v8) at (2,-1) {};
        \node [normal, label=above right:$u_4$] (v4) at (0.5,-1.95) {};
        \node [normal, label=above:$u_5$] (v5) at (-2,-1) {};
        \node [normal, label=above right:$u_6$] (v6) at (-2.5,0.5) {};
        \node [normal, label=above right:$u_7$] (v7) at (-2,2) {};
        \draw [thick] (v1) edge (v2);
        \draw [thick] (v1) edge (v3);
        \draw [thick] (v1) edge (v4);
        \draw [thick] (v5) edge (v1);
        \draw [thick] (v1) edge (v6);
        \draw [thick] (v7) edge (v1);
        \draw [thick] (v8) edge (v1);
    \end{tikzpicture}
    \caption{The star graph $S_7$}
    \label{fig:starS7}
\end{figure}

If Alice starts she will play $v$ and the game ends with $|I| = 1$. If Bob starts he will play a vertex $u_i$. Alice cannot play $v$ so must play a vertex $u_j$ such that $i\neq j$. The game will end with $|I| = 7$. Thus $I_A(S_7)=1$ and $I_B(S_7)=7$.

\begin{definition}[Independence number]
    The \textit{independence number} of a graph $G$, denoted $\alpha(G)$, is the size of the independent set in $G$.
\end{definition}

A graph has no independent sets larger than it's independence number. Consequently, when playing the competition-independence game the size of the independent set formed is no larger than the  independence number. This means the competition-independence number is bounded above by the independence number. Hence, we get theorem \ref{thm:indgameup}.

\begin{theorem}\label{thm:indgameup}
    For a graph $G$, $I_A(G)\leq \alpha(G)$.    
\end{theorem}

%todo add bound

%If 
%
%It was shown in Borg 2010 \cite{borg2010sharp} that for a graph $G$ with maximal degree $k$ and $n$ vertices 
%%
%\begin{equation}
%\label{equ3}
%\alpha(G) \leq n - \left\lceil\frac{n-1}{k}\right\rceil
%\end{equation}
%
%Further the authors show that this bound is sharp. 
%
%$I_A$ is bounded above by the independence number. As the bound in equation \ref{equ3} is sharp the independence number of the class of graphs with maximal degree $k$ is bounded below by $n - \left\lceil(n-1)/k\right\rceil$. Hence for the class of graphs $\mathcal{C}$ with maximal degree $k$ and $n$ vertices 
%
%\[\gamma_g(\mathcal{C}) = n -\left\lceil\frac{n-1}{k}\right\rceil\] 












