\chapter{Graph Domination}\label{chpt:domSet}

\section{Introduction}

We introduced dominating sets in chapter \ref{chpt:into} with the dinner party problem. The size of a minimal dominating set, is in some sense, a measure of how unconnected a graph. A graph with a low domination number has high connectivity, and a graph with a high domination number has low connectivity. For example any complete graph $K_n$ has domination number 1, and the graph that is just the set of $n$ independent vertices has domination number $.$. More formally we define  both dominating sets and the dominating number as follows.

\begin{definition}
    Let $G$ be a graph. A Dominating set, $D$, of $G$ is a subset of $V(G)$ such that every vertex in $V(G)$ is adjacent to at least one vertex in $D$.
\end{definition} 
\begin{definition}
    Let $G$ be a graph. The Domination number, $\gamma(G)$, of $G$ is the minimum size of a dominating set in $G$.
\end{definition}

 The dominating game was first introduced in form we are interested in by Bo\v{s}tjan Bre\v{s}ar, Sandi Klav\v{z}ar and Douglas RallF in 2010 \cite{BrKlRa2010}. There have been earlier versions of a dominating game, though these games were considered with minimising the dominating number in an induced orientation of a graph. We explore the game where the players build a dominating set directly and where Alice is trying to minimise its size. 

%\begin{definition}
%    Independent set, maximum independent set, independence number $\alpha(G)$
%\end{definition}

\section{Dominating game}

We define the dominating game as follows. Let $G$ be a graph, $t$ be a target score and initialise $D=\emptyset$. On alternating turns, beginning with Alice, Alice and Bob add vertices to $D$ one at a time. The game stops when $D$ forms a dominating set in $G$. The score of the game is $s=|D|$. Alice wins if $s<t$ and Bob wins otherwise. %todo add the requirement that on each turn the D must grow?

\begin{definition}(game domination number)
    Let $G$ be a graph. The game dominating number, $\gamma_g(G)$ is the minimum target score such that Alice has a winning strategy.
\end{definition} 

%\begin{lemma}
%    Let $G$ be a graph. 
%        
%    \[\gamma(G) \geq \alpha(G)\]
%\end{lemma}
%
%\begin{proof}
%    Let $X$ be a minimum dominating set in some graph $G=(V,E)$. By definition of dominating set vertex in $V$ is adjacent to at least one vertex in      
%\end{proof}
%    
%Recall that $\chi(G)$ is the chromatic number of the graph $G$.
%
%\begin{theorem} [Willis 3.1 \cite{Willis2011BoundsFT}] \label{willis3.1}
%    For any graph $G = (V,E)$ 
%
%    \[\alpha(G) \leq \frac{ \left | {V} \right |}{\chi(G)}\]
%\end{theorem}
%
%Recall that $\Delta(G)$ is the maximum degree of any vertex in $G$.
%
%\begin{theorem} [Balakrishnan 10.3.2 \cite{balakrishnan2012}] \label{balakrishnan10.3.2}
%    For any graph $G$ with $n$ vertices, 
%    
%    \[ \left\lceil {\frac{n}{1+\Delta(G)}} \right\rceil \leq \gamma(G) \leq n - \Delta(G)\]    
%\end{theorem}
    
For some graph,$G$, $\gamma(G)$ is the size of the smallest domination set and hence $\gamma(G)\leq\gamma_g(G)$. Hence $\gamma(G)$ is the trivial lower bound for the game domination number. Note theorem \ref{oreDomUpper}, it is a well known result that $\gamma(G) \leq n/2$, we will show $n/2$ is a lower bound for the game domination number. But first a couple of theorems.

\begin{theorem}[Ore 1962 \cite{oysteinore1962}] \label{oreDomUpper} 
    For any connected graph $G$ with $n$ vertices,     
    \[\gamma(G) \leq \frac{n}{2}\]
\end{theorem}

%TODO prove that Ore is tight

\begin{theorem}\label{thm:domlow}
    Let $\GG$ be a family of graphs and $\gamma(\GG)$ tight upper bound for the domination number of $\GG$. That is for all $G\in\GG$ $\gamma(G)\leq\gamma(\GG)$ and there exists some $G\in\GG$ such that $\gamma(G)=\gamma(\GG)$. Then,
    \[\gamma(\GG) \leq  \gamma_g(\GG)\]
\end{theorem}

\begin{proof}
    Let $\GG$ be a family of graphs and $G\in\GG$ a graph such that $\gamma(G) = \gamma(\GG)$.
    Therefore for $G$ we are unable to find a dominating set with less than $ \gamma(\GG) $ vertices.
    Therefore there cannot be a winning strategy for Alice with a score less than $\gamma(\GG)$.
    Therefore $\gamma(\GG) \leq \gamma_g(\GG)$
\end{proof}

%add table of classes of graphs

\begin{corollary} \label{thm:gamedomLow}
    Let $G$ be a connected graph with $n$ vertices, such that $n \geq 4$. Then,    
    \[  \frac{n}{2} \leq \gamma_g(G)  \]    
\end{corollary}

\begin{proof}
    Let $G$ be a graph with $n$ vertices.
    
    Consider the complete graph with $2$ vertices $K_2$, a dominating set in $K_2$ is a single vertex, and hence $\gamma(K_2)=n/2=1$. Therefore $n/2$ is a tight upper bound for $\gamma(G)$. 
    
    Therefore by theorem \ref{thm:domlow} $ n/2 \leq \gamma_g(G)$.
\end{proof}

Corollary \ref{thm:gamedomLow} does not say that for all graphs with domination number less than $n/2$ Alice will always loss but rather target score needed to ensure Alice will always win on a connected graph is greater than $n/2$.

%TODO insert disucssion
%C \ref{thm:gamedomLow} is also proved in Alona, Baloghc, Bollobas, and Szabo 2002 \cite{AlBABoSz2002}.

An upper bound for the game domination number for a class of graphs is a target score that guarantees Alice has a winning strategy. Consider some graph $G$, the vertex set $V(G)$ of $G$ is a dominating set and hence we have the trivial upper bound $\gamma_g(G) \leq |V(G)|$. However we can do better.

\begin{theorem}\label{thm:gamedomup}
    Let $G$ be a graph and $\gamma(G)$ the dominating number of $G$, then 
    \[\gamma_g(G)<2\gamma(G)\]
\end{theorem}
\begin{proof}
    Let $G$ be a graph and $X\subseteq V(G)$ a dominating set such that $|X| = \gamma(G)$. Suppose Bob has just chosen a vertex and $D$ is the current dominating set. Alice's strategy is too pick any vertex, $v$, in $X$ that $v\notin D$. After $\gamma(G)$ turns the game must have ended as $X$ is a dominating set. On each turn both Alice and Bob added one vertex to $D$, hence the size of $D$ is at most $2\gamma_g$.
\end{proof}
%hence bounds for game relie on bounds for dom num

\begin{theorem}
    Let $T$ be a tree with $n$ vertices. Then,    
    \[ \gamma_g(T) \leq \left\lceil \frac{2n}{3} \right\rceil\]
\end{theorem}
 
%\begin{proof}
%    Let $T$ be a tree with $n$ vertices. On Alice's first move she picks the vertex, $v$, with maximal degree. Initialise $A=N[v]$ to denote the activated vertices. 
%
%    Suppose Bob has just picked a vertex $u$ and let $D$ be the current dominating set in $T$. The winning strategy for Alice is the greedy strategy as follows.    
%    
%    Let $U=V(T)\ba D$ the unchosen vertices in $T$. Alice uses algorithm \ref{alg:greedy} to update $A$ and choose the next vertex.
%        
%    \begin{algorithm}[h]
%        \caption{Greedy strategy}
%        \label{alg:greedy}
%        \begin{algorithmic}[1] 
%            \State $A\gets A \cup N(u)$
%            \State let $x\in U$ be the vertex that maximises $|N(x)\ba A|$
%            \State $A\gets A \cup N(x)$            
%            \State Choose $x$            
%        \end{algorithmic}
%    \end{algorithm}
%    
%   worst case path graph requires twice the minimum of the path graph???
%   
%   with no opponent this will give n/3 thus at worst with the opponent it will take 2n/3 
%       
%    At worst Alice will add two vertices to 
%    %TODO finish this proof
%\end{proof}

\section{The $(a,b)$--dominating game}

The natural extension of the dominating game is to allow Alice and Bob to select more than one vertex on each turn. We define this as the $(a,b)$--dominating game when Alice is allowed to select $a$ vertices per turn and Bob $b$ vertices.

\begin{definition}[game domination number]
    Let $G$ be a graph and $a,b>1$ integers. The game dominating number $\gamma_g(G;a,b)$ is the smallest target score such that Alice has a winning strategy when playing the $(a,b)$--dominating game.
\end{definition}

\begin{theorem} %TODO prove this theorem
    Let $G$ be a graph with $n$ vertices and $a,b>1$ integers. Then,
%    \[\gamma_{g}(G;1,b) \geq (b+1)\gamma(G) \]    
    \[\gamma_{g}(G;a,b) \leq \frac{a+b}{a}\gamma(G)\]
\end{theorem}
\begin{proof}
    This proof proceeds in much the same way as the proof of theorem \ref{thm:gamedomup}.
    
   Let $G$ be a graph and $X\subseteq V(G)$ a dominating set of $G$ such that $|X| = \gamma(G)$. Suppose Bob has just chosen $b$ vertices and $D$ is the current dominating set. Alice's strategy is to pick any $a$ vertices, $v_1,\dots,v_a$, in $X$ that each $v_i\notin D$. After $\frac{\gamma(G)}{a}$ rounds the game must have ended as $X$ is a dominating set and at least $a$ elements of $X$ are chosen each round. In each round Alice adds $a$ vertices to $D$ and Bob adds $b$ vertices to $D$, hence the size of $D$ is at most $(a+b)\frac{\gamma(G)}{a}$.
\end{proof}













