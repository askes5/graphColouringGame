\chapter{Introduction}\label{chpt:into}
%todo what is a graph
A graph, $G$, is a collection of vertices, $V(G)$, and edges, $E(G)$. Where the edges, represented as pairs of vertices, define a relation on the vertices. For example in figure \ref{fig:k3} we have three edges $(a,b)$, $(a,c)$, and $(b,c)$. There are two main types of graphs directed where the direction of the edge matters ($(a,b)\neq (b,a)$) and undirected where the direction doesn't matter ($ab=(b,a)$).

\begin{figure}[h]
    \centering
\begin{tikzpicture}[scale=2]
\node[normal, label=below:$a$] (v1) at (-1.5,-1) {};
\node[normal, label=below:$b$] (v3) at (-0.5,-1) {};
\node[normal, label=below:$c$] (v2) at (-1,0) {};  
\draw (v1) edge (v2);
\draw  (v2) edge (v3);
\draw  (v3) edge (v1);
\end{tikzpicture}
    \caption{An undirected graph with three vertices and three edges}
\label{fig:k3}
\end{figure}
   
We introduce the idea of a graph game by considering the \textit{Dinner Party Problem}.
Suppose Alice is hosting a party. All of the guest are mingling happy. However the guests are hungry and they demand to be feed but they are lazy will not move to collect food. To serve the food Alice decides to pass out platters to guests. When a guest receives a platter they take a piece and everyone within arms reach take a piece. As the food is expensive Alice wants to minimise the amount of platters handed out. What is the minimum number of platters that Alice needs to feed everybody? 

We can define a graph, $G$, with guests as vertices and edges between two guests if when either of them are given a plate they will pass food between them. Minimising the number platters is now a matter of finding a minimum dominating set.

Alice decides to host a second party after the success of the first, but she decides to enlist the help of her friend Bob to help distribute the platters. Alice and Bob take turns handing out platters. As before Alice tries to minimise the number of platters, Bob on the other hand has no such desire and will pass out platters as he sees fit. If Alice has the same number of platters as the first party it is conceivable that Bob can force some people to go hungry. The problem now is what is the smallest number of platters that Alice need to ensure that everyone will get feed? 

If we define a graph as above, minimizing platters is a matter of finding the best strategy for Alice in the dominating game. 
As an aside, it is the intention to imply that Bob is trying to ruin Alice's party but rather he has his own criteria for distributing platters, perhaps he is trying to get everyone their favourite food with no regard to cost. We explore the idea of game dominating numbers in Chapter \ref{chpt:domSet}.

It is easy to see how this concept could be applied to other situations. For example bombing rail roads, where the object is to minimise bombs used but your allies are unreliable. Or Examples include a measure of robustness in network infrastructure, scheduling, and register allocation.  

When considering graph games we consider them to be played by two players Alice and Bob, where Alice has the first move. Later in this report we explore various different games on graphs. We introduce upper and lower bounds for graph some classes of graphs and provide details on popular strategies when playing these games.   

 %todo why is this iteresting, applications

%todo What this this report about

%todo where has this come from

  
%This work has many applications in computer science and operations research, including flight scheduling, bandwidth allocation and register allocation, as well as deep theoretical interest%note this is plagrisum
    
A note on notation. For $\CC$ a family of graphs, and $\varphi_g(G)$ some property of a graph $G \in \CC$. Then $\varphi_g(\CC)$ is the smallest $k$ such that for every graph $H\in\CC$ $\varphi_g(H)\leq k$.   

%todo add summery of results
    
    
    
    
    
    
    
    
    
    
    
    
    
    
    
    
    
    
    
    
    