\chapter{Introduction}\label{chpt:into}
%todo what is a graph
A graph, $G$, is a set of vertices, $V(G)$, and edges, $E(G)$. Where the set of edges defines a relation on $V(G)$, and are represented as pairs of vertices. For example in figure \ref{fig:k3} we have three edges $(a,b)$, $(a,c)$, and $(b,c)$. There are two main types of graphs we consider, directed graphs where the direction of the edge matters ($(a,b)\neq (b,a)$) and undirected graphs where the direction doesn't matter ($(a,b)=(b,a)$).

\begin{figure}[h]
    \centering
\begin{tikzpicture}[scale=2]
\node[normal, label=below:$a$] (v1) at (-1.5,-1) {};
\node[normal, label=below:$b$] (v3) at (-0.5,-1) {};
\node[normal, label=below:$c$] (v2) at (-1,0) {};  
\draw (v1) edge (v2);
\draw  (v2) edge (v3);
\draw  (v3) edge (v1);
\end{tikzpicture}
    \caption{An undirected graph with three vertices and three edges}
\label{fig:k3}
\end{figure}
   
We introduce the idea of a graph game by considering the \textit{Dinner Party Problem}.
Suppose Alice is hosting a party and all the guests are mingling happily. However, the guests are hungry and need to be feed, but they are lazy will not move to collect food. To serve the food Alice decides to pass out platters to guests. When a guest receives a platter they take a piece and pass food to everyone within arm's reach. As the food is expensive Alice wants to minimise the total number of platters handed out. What is the minimum number of platters that Alice needs to feed everybody? To see this, we can define a graph, $G$, with guests as vertices and an edge between two guests if when either of them are given a platter they will pass food between them. Minimising the number platters is now a matter of finding a set of vertices, $D$ in $G$ such that every vertex in $V(G)$ is incident to some vertex in $D$.

After the success of the first party, Alice decides to host a second party. She decides to enlist the help of her friend Bob to help distribute the platters. Beginning with Alice, on alternating turns Alice and Bob hand out a platter. As before Alice tries to minimise the number of platters, Bob on the other hand has no such desire and will pass out platters however he sees fit. Bob may not be intending to ruin Alice's party but rather he may have some criteria for handing out platters that is unknown. If Alice has the same number of platters as the first party Bob could quite easily force someone to go hungry by always picking a person that will feed the least number of people. Hence, Alice needs more platters than the first party to feed everyone. The \textit{Dinner Party Problem} is what is the smallest number of platters that Alice needs to guarantee that everyone will always get feed? If we define a graph as in the first party, finding a minimum number of platters is a matter of finding an optimal strategy for Alice to hand out a minimum number of platters. We further explore this idea in the context of the domination game and the game domination number in Chapter \ref{chpt:domSet}.

It is easy to see how this concept could be applied to other situations. For example, in wartime a nation may be trying to destroy railroads, using the minimum number bombs possible, but their allies are unreliable and cannot to relied upon to share the game goals. Other Examples include a measure of robustness in network infrastructure, scheduling, and register allocation.  

In this report %todo report?
we explore various different graph games with hostile partners. We also introduce upper and lower bounds for some classes of graphs when playing the domination game and the colouring game and provide details on popular strategies when playing these games.



 %todo why is this iteresting, applications

%todo What this this report about

%todo where has this come from

  
%This work has many applications in computer science and operations research, including flight scheduling, bandwidth allocation and register allocation, as well as deep theoretical interest%note this is plagrisum
    
A note on notation. If $\CC$ is a family of graphs, and $\varphi_g(G)$ some property of a graph $G \in \CC$. Then $\varphi_g(\CC)$ is the smallest $k$ such that for every graph $H\in\CC$ $\varphi_g(H)\leq k$. Finally, when considering a graph game we always consider it to be played by two players Alice and Bob, where Alice has the first move.     

%todo add summery of results
    
    
    
    
    
    
    
    
    
    
    
    
    
    
    
    
    
    
    
    
    