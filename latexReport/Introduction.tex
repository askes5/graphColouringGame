\chapter{Introduction}\label{chpt:into}
%todo what is a graph
A graph, $G$, is a set of vertices, $V(G)$, and edges, $E(G)$. Where the set of edges defines a relation on $V(G)$, and are represented as pairs of vertices. For example in figure \ref{fig:k3} we have three edges $(a,b)$, $(a,c)$, and $(b,c)$. There are two main types of graphs we consider, directed graphs where the direction of the edge matters ($(a,b)\neq (b,a)$) and undirected graphs where the direction doesn't matter ($(a,b)=(b,a)$).

\begin{figure}[h]
    \centering
\begin{tikzpicture}[scale=2]
\node[normal, label=below:$a$] (v1) at (-1.5,-1) {};
\node[normal, label=below:$b$] (v3) at (-0.5,-1) {};
\node[normal, label=below:$c$] (v2) at (-1,0) {};  
\draw (v1) edge (v2);
\draw  (v2) edge (v3);
\draw  (v3) edge (v1);
\end{tikzpicture}
    \caption{An undirected graph with three vertices and three edges}
\label{fig:k3}
\end{figure}
   
%ubiqity of graphs 
Graphs are everywhere. Though they may not be immediately apparent. 
Graphs help us understand data and structures in a way that is easier to visualise and analyse. An example from recent events is COVID--19, and more generally infectious diseases.  We can build a graph by considering infected people as vertices in the graph, and we join them with an edge if the disease spread between them. %todo insert picture of spread
From such a graph we can easily determine things like, who is the most infectious (who has the most infected neighbours), and where is community transmission is occurring (a peice of the graph that is disconnected from the rest represents an unknown origin).
     
%graph algorithms   
A graph algorithm is a set of instructions that define some procedure on a graph. 
For example an algorithm for finding a minimum spanning tree is as simple as picking the next smallest edge such that each edge does not create a cycle in the tree.  

Suppose you are charged with laying fibre optic cable in a neighbourhood. Your goal is to use the least amount of cable possible while still connecting all the houses. We can form a graph with the houses as vertices and an edge between two houses if it is possible to lay cable between them. We then assign each edge a weight representing the cost of laying cable between the two houses. The smallest amount of cable need is now a matter of finding a minimum spanning tree in this graph. 
   
We introduce the idea of a graph game by considering the \textit{Dinner Party Problem}.
Suppose Alice is hosting a party and all the guests are mingling happily. However, the guests are hungry and need to be feed, but they are lazy will not move to collect food. When serving the food, Alice places platters of food around the room near the guests. A guest will take some food and be feed if a platter is placed within arm's reach. The food is expensive. Hence, Alice wants use the smallest number of platters possible. The problem is then, what is the minimum number of platters that Alice needs to feed everybody? 

%To see this, we can define a graph, $G$, with guests as vertices and an edge between two guests if when either of them are given a platter they will pass food between them. Minimising the number platters is now a matter of finding a set of vertices, $D$ in $G$ such that every vertex in $V(G)$ is incident to some vertex in $D$.

After the success of the first party, Alice decides to host a second party. To alleviate the pressure of hosting she decides to hire a caterer, Bob. As before the platters are placed around the room and a guest will be take some food if there is a platter within arms reach. Staring with Alice, on alternating turns Alice and Bob place a platter all the guests are feed. As before Alice tries to minimise the number of platters. Bob on the other hand makes a profit for every platter and so will try to maximise the number of platters. Every platter Bob places must feed at least one new person, otherwise Alice would notice Bob's scheme.
If Bob was to always place a platter that will feed the least number of people then the total number of platters places would be greater than the first party. Hence, Alice requires some strategy to minimise the platters placed. The \textit{Dinner Party Problem} is then, what is the smallest number of platters that Alice can guarantee will always feed everyone? 
%If we define a graph as in the first party, finding a minimum number of platters is a matter of finding an optimal strategy for Alice to hand out a minimum number of platters. 
We further explore this idea in the context of the domination game and the game domination number in Chapter \ref{chpt:domSet}.

It is easy to see how this concept could be applied to other situations. For example, in wartime a nation may be trying to destroy railroads, using the minimum number bombs possible, but their allies are unreliable and cannot to relied upon to share the game goals. Other Examples include a measure of robustness in network infrastructure, scheduling, and register allocation.  

In this report %todo report?
we explore various different graph games with hostile partners. We also introduce upper and lower bounds for some classes of graphs when playing the domination game and the colouring game and provide details on popular strategies when playing these games.



%todo why is this iteresting, applications

%todo What this this report about

%todo where has this come from

  
%This work has many applications in computer science and operations research, including flight scheduling, bandwidth allocation and register allocation, as well as deep theoretical interest%note this is plagrisum
    
A note on notation. If $\CC$ is a class of graphs, and $\varphi_g(G)$ some property of a graph $G \in \CC$. Then $\varphi_g(\CC)$ is the smallest $k$ such that for every graph $H\in\CC$ $\varphi_g(H)\leq k$. Finally, when considering a graph game we always consider it to be played by two players Alice and Bob, where Alice has the first move.  

A round is a play of both alice and Bob. A turn is a play... %todo terms and definations
   %todo class of graphs?


%todo add summery of results
    
    
    
    
    
    
    
    
    
    
    
    
    
    
    
    
    
    
    
    
    